\documentclass[a4paper,12pt]{article}

%%% Работа с русским языком
\usepackage{cmap}					          % поиск в PDF
\usepackage[T2A]{fontenc}			      % кодировка
\usepackage[utf8]{inputenc}               % кодировка исходного текста
\usepackage[english, russian]{babel}   % локализация и переносы

\usepackage{minted}

%%% Страница 
\usepackage{extsizes} % Возможность сделать 14-й шрифт
\usepackage{geometry}  
\geometry{left=20mm,right=20mm,top=25mm,bottom=30mm} % задание полей текста

\usepackage{titleps}      % колонтитулы
\newpagestyle{main}{
	\setheadrule{.4pt}                      
	\sethead{\CourseName}{}{\hyperlink{intro}{\;Назад к содержанию}}
	\setfootrule{.4pt}                       
	\setfoot{\CourseDate \; ФПМИ МФТИ}{}{\thepage} 
}      
\pagestyle{main}    % Устанавливает контитулы на странице

\usepackage{tocloft}

\advance\cftsecnumwidth 0.5em\relax
\advance\cftsubsecindent 0.5em\relax
\advance\cftsubsecnumwidth 0.5em\relax

%%%  Текст
\setlength\parindent{0pt}         % Устанавливает длину красной строки 0pt
\sloppy                                        % строго соблюдать границы текста
\linespread{1.3}                           % коэффициент межстрочного интервала
\setlength{\parskip}{0.5em}                % вертик. интервал между абзацами
%\setcounter{secnumdepth}{0}                % отключение нумерации разделов
%\setcounter{section}{-1}         % Чтобы сделать нумерацию лекций с нуля
\usepackage{multicol}				          % Для текста в нескольких колонках
%\usepackage{soul}
\usepackage{soulutf8} % Модификаторы начертания


%%% Гиппер ссылки
\usepackage{hyperref}
\usepackage[usenames,dvipsnames,svgnames,table,rgb]{xcolor}
\hypersetup{				% Гиперссылки
	unicode=true,           % русские буквы в раздела PDF\\
	pdfstartview=FitH,
	pdftitle={Заголовок},   % Заголовок
	pdfauthor={Автор},      % Автор
	pdfsubject={Тема},      % Тема
	pdfcreator={Создатель}, % Создатель
	pdfproducer={Производитель}, % Производитель
	pdfkeywords={keyword1} {key2} {key3}, % Ключевые слова
	colorlinks=true,       	% false: ссылки в рамках; true: цветные ссылки
	linkcolor=blue,          % внутренние ссылки
	citecolor=green,        % на библиографию
	filecolor=magenta,      % на файлы
	urlcolor=NavyBlue,           % на URL
}

\usepackage{todonotes}


%%% Для формул
\usepackage{amsmath}          
\usepackage{amssymb}


%%%%%% theorems
\usepackage{amsthm}  % for theoremstyle

\theoremstyle{plain} % Это стиль по умолчанию, его можно не переопределять.
\newtheorem{theorem}{Теорема}[section]
\newtheorem{prop}[theorem]{Утверждение}
\newtheorem{lemma}{Лемма}[section]
\newtheorem{sug}{Предположение}[section]

\theoremstyle{definition} % "Определение"
\newtheorem{Def}{Определение}
\newtheorem{corollary}{Следствие}[theorem]
\newtheorem{problem}{Задача}[section]

\theoremstyle{remark} % "Примечание"
\newtheorem*{nonum}{Решение}
\newtheorem*{definition}{Def}
\newtheorem*{example}{Пример}
\newtheorem*{note}{Замечание}


%%% Работа с картинками
\usepackage{graphicx}                           % Для вставки рисунков
\graphicspath{{images/}{images2/}}        % папки с картинками
\setlength\fboxsep{3pt}                    % Отступ рамки \fbox{} от рисунка
\setlength\fboxrule{1pt}                    % Толщина линий рамки \fbox{}
\usepackage{wrapfig}                     % Обтекание рисунков текстом
\graphicspath{{images/}}                     % Путь к папке с картинками

\newcommand{\drawsome}[1]{            % Для быстрой вставки картинок
    \begin{figure}[h!]
            \centering
            \includegraphics[scale=0.7]{#1}
            \label{fig:first}
    \end{figure}
}
\newcommand{\drawsomemedium}[1]{
    \begin{figure}[h!]
            \centering
            \includegraphics[scale=0.45]{#1}
            \label{fig:first}
    \end{figure}
}
\newcommand{\drawsomesmall}[1]{
    \begin{figure}[h!]
            \centering
            \includegraphics[scale=0.3]{#1}
            \label{fig:first}
    \end{figure}
}


%%% облегчение математических обозначений
\newcommand{\R}{\mathbb{R}}
\newcommand{\N}{\mathbb{N}}
%\newcommand{\C}{\mathbb{C}}             % команда уже определена где-то)
\newcommand{\Z}{\mathbb{Z}}
\newcommand{\E}{\mathbb{E}}
\newcommand{\brackets}[1]{\left({#1}\right)}      % автоматический размер скобочек
% Здесь можно добавить ваши индивидуальные сокращения 

%%%%%%%%%%%%%%%%%%%%%%%%%%%

\usepackage[utf8]{inputenc} % Включаем поддержку UTF8
\usepackage[russian]{babel}  % Включаем пакет для поддержки русского языка
\usepackage{mathtools}
\usepackage{amssymb}
\usepackage{amsthm}
\usepackage{graphicx}
\usepackage{subcaption}\usepackage{caption}
\usepackage{svg}

\usepackage{listings}
\usepackage{xcolor}

\newcommand{\listingsttfamily}{\fontfamily{NotoSansMono-TLF}\big}
\newcommand*\lsin{\lstinline[columns=fixed, basicstyle=\fontsize{13}{13}\ttfamily]}

\lstset{
  language = [11]C++,
  backgroundcolor=\color{black!5}, % set backgroundcolor
  keywordstyle = \color{blue},
  columns = flexible
}

\usepackage{xpatch}
\xpatchcmd{\minted}{\VerbatimEnvironment}{\VerbatimEnvironment\let\itshape\relax}{}{}
\usepackage{tcolorbox}
\usepackage{etoolbox}
\BeforeBeginEnvironment{minted}{\begin{tcolorbox}[
                  leftrule=0.5pt,
                  rightrule=0.5pt,
                  toprule=0.5pt,
                  bottomrule=0.5pt,
                  boxsep=0pt,
                  left=0pt,
                  right=0pt,
                  top=-1.7pt,
                  bottom=-1.7pt
                  ]}%
\AfterEndEnvironment{minted}{\end{tcolorbox}}%

\definecolor{bg}{rgb}{0.95,0.95,0.95}

\newminted[cminted]{c}{
  numbersep=5pt,
  framesep=2mm,
  baselinestretch=1.2,
  bgcolor=bg,
  linenos}

% https://github.com/heia-fr/pygments-arm
\newminted[asmminted]{ARM}{ 
  numbersep=5pt,
  framesep=2mm,
  baselinestretch=1.2,
  bgcolor=bg,
  linenos}

\newminted[bminted]{bash}{
  numbersep=5pt,
  framesep=2mm,
  baselinestretch=1.2,
  bgcolor=bg,
  linenos}

\newmintinline[asmmint]{ARM}{}
\newmintinline[cppmint]{cpp}{}
\newmintinline[cmint]{c}{}
\newmintinline[bmint]{bash}{}
                  
\newcommand\round[1]{\left[#1\right]}

\tolerance=1
\emergencystretch=\maxdimen
\hyphenpenalty=10000
\hbadness=10000


%%% Всю шаблонную информацию можно менять тут
\newcommand{\FullCourseNameFirstPart}{\so{АРХИТЕКТУРА КОМПЬЮТЕРОВ И}}
\newcommand{\FullCourseNameSecondPart}{\so{ОПЕРАЦИОННЫЕ СИСТЕМЫ}}
\newcommand{\SchoolName}{ФПМИ}
\newcommand{\TrackName}{ПМИ}
\newcommand{\SemesterNumber}{III}
\newcommand{\LecturerInitials}{Андреев Александр Николаевич}
\newcommand{\AutherInitials}{Кирилл Афентьев}
\newcommand{\VKLink}{https://yandex.ru}
\newcommand{\GithubLink}{https://github.com/MIPT-Group/Lectures_Tex_Club}

\newcommand{\CourseName}{АКОС} %  Используется в преамбуле для создания названия предмета в верхнем контитуле   
\newcommand{\CourseDate}{Осень 2022}           %  Используется в преамбуле для создания даты в нижнем контитуле

%\includeonly{lectures/lect01,lectures/lect02}  % Чтобы скомпилировать только часть лекций

\begin{document}
    \begin{titlepage}
	\clearpage\thispagestyle{empty}
	\centering
	
	\textit{Федеральное государственное автономное учреждение \\
		высшего образования}
	\vspace{0.5ex}
	
	\textbf{Московский физико-технический институт
    \\
    (национальный исследовательский университет)}
	\vspace{20ex}
	
	\rule{\linewidth}{0.5mm}
	{\textbf{\FullCourseNameFirstPart}}
	\\
	{\textbf{\FullCourseNameSecondPart}}
	\rule{\linewidth}{0.5mm}
	
	\SemesterNumber\ СЕМЕСТР
	\\
	Физтех-школа: \textit{\SchoolName}
	\\
	Направление: \textit{\TrackName}
	\vspace{1ex}
	
	\begin{figure}[!ht]
		\centering
		\includegraphics[width=0.5\textwidth]{mipt}
		\label{fig:mipt}
	\end{figure}
\begin{flushright}
	\noindent
	%\\
	%\href{\OverleafLink}{\textit{Проект на overleaf}}
	\\
\end{flushright}
	
	\vfill
	Долгопрудный, \CourseDate\ год.
	\pagebreak
	
\end{titlepage}
    \newpage
    \hypertarget{intro}{}
    \tableofcontents

    \section{Вводная лекция}
  \subsection{Операционная система}
  Операционная система -- абстракция, которая связывает различные компоненты компьютера и пользовательские программы.
  
  \subsection{Из каких компонент состоит компьютер?}
  \begin{itemize}
    \item Центральный процессор (CPU или ЦП)
    \item Чипсет и материнская плата
    \item Оперативная память (Random Access Memory = RAM)
    \item Накопители (HDD, SSD, NVMe)
    \item Аудиокарта
    \item Сетевая карта
    \item GPU
    \item Шина (PCI, I2C, ISA)
  \end{itemize}
  
  \subsubsection{Процессор}
  \begin{itemize}
    \item Исполняет команды или \textit{инструкции}
    \item Регистры -- самые быстрые доступные ячейки памяти
    \item Регистры определяют разрядность процессора
    \item Операндами могут быть либо константы, либо регистры, либо ссылки на память
  \end{itemize}
  
  \subsubsection{Оперативная память}
  \begin{itemize}
    \item Random Access Memory
    \item Адресное пространство -- непрерывный массив байт от $0$ до $2^N$, где $N$ -- разрядность процессора ($64$ бита)
    \item В реальности процессоры на текущий момент обычно адресуют не более $48$ бит ($256$ терабайт)
    \item Инструкции процессора расположены также в RAM -- архитектура Фон-Неймана
  \end{itemize}
  
  Сейчас оперативная память работает значительно медленнее процессора (доступ к RAM занимает несколько десятков инструкций процессора). Поэтому внутри процессора есть несколько уровней своей ``оперативной памяти'': L1, L2, L3. Они устроены немного иначе, чем оперативная память, и стоят очень дорого. Если запрашивается доступ к 1 байту, а затем к следующему байту, то второе считывание будет сделано не из оперативной памяти, а из кэша (L1/L2/L3, в зависимости от их наполнения). О том, почему есть несколько уровней кэша, будет рассказано в следующих лекциях.
  Из-за существования кэшей, нам выгодно, чтобы данные лежали ``рядом'' в памяти. Один из примеров: \href{https://levelup.gitconnected.com/c-programming-hacks-4-matrix-multiplication-are-we-doing-it-right-21a9f1cbf53}{ускорение умножения матриц}.
  
  \subsection{Немного ассемблера}
  Ассемблер -- это вид для человека, эти команды -- не процессорные инструкции. На современных процессорах Intel длина инструкции обычно занимает от 1 до 8 байт. В этом курсе будет рассмотрена только архитектура x86. В инструкцию зашивается вся нужная информация: используемые константы, используемые адреса памяти и т.д. Подробнее об этом будет рассказано позже.

\begin{lstlisting}[style=asm]
mov rax, qword ptr [rax]
add rax, 2
mov rbx, 1
add rax, rbx
\end{lstlisting}

\lsin{rax}, \lsin{rbx} -- это регистры процессора. Всего различных регистров общего назначения 16. \newline

Первая инструкция в этом коде берёт адрес регистра \lsin{rax}, считывает его содержимое, и записывает в него же, в \lsin{rax}. \newline
Вторая команда прибавляет к содержимому \lsin{rax} $2$. \newline
Третья команда записывает в \lsin{rbx} $1$. \newline
Четвертая команда прибавляет \lsin{rbx} к \lsin{rax}.
        
    \subsection{Мультизадачность}
    \begin{itemize}
      \item Мультизадачность -- способность системы исполнять несколько задач (процессов) одновременно
      \item Cooperative multitaksing -- процессы добровольно передают управление друг другу
      \item Preemptive multitasking -- процессы вытесянются ОС каждые несколько миллисекунд
    \end{itemize}
    
    Минус cooperative multitasking: если процесс завис, то он не передаст управление дальше, остается только reset.\newline
    Первая Windows, в которой появился multitasking, это Windows 95, до этого был singleprocess MS-DOS.
    
    \subsubsection{Суперскалярность}
    \begin{itemize}
      \item Параллелизм уровня инструкций
      \item Если две инструкции независимы друг от друга, их можно выполнить параллельно
      \item Каждая инструкция состоит из нескольких этапов: fetch, decode, execute, memory access, register write back
      \item CPU pipeline
    \end{itemize}
    Пример процессора без суперскалярности: российский Эльбрус, в котором одна инструкция процессора содержит несколько операций, которые выполняются параллельно. Такой принцип называется VLIW -- Very Long Instruction Word.
    
    \subsubsection{CPU pipeline}
    \begin{figure}[h!]
  \includegraphics[width=\linewidth]{/Users/user/Downloads/cpu_pipeline.png}
  \caption{CPU pipeline}
  \label{fig:pipeline}
\end{figure}
    
    \subsubsection{Мультипроцессорность}
    \begin{itemize}
      \item Тактовая частота процессоров не растет примерно с 2005 года
      \item Поэтому современные процессоры обычно имеют несколько ядер
      \item \textit{Планировщик (scheduler)} ОС для каждого ядра процессора в каждый момент времени решает какой процесс будет запущен
      \item Возникают проблемы синхронизации
    \end{itemize}
    
    \subsection{Системные вызовы}
    \begin{itemize}
      \item Системные вызовы -- это интерфейс операционной системы для процессов
      \item ABI = application binary interface
      \item SystemV ABI
    \end{itemize}
    Системный вызов -- это очень дорогая операция. У каждой операционной системы свой ABI.
    
    \subsection{POSIX}
    \begin{itemize}
      \item Portable Operating System Interface
      \item Стандарт, описывающий интерфейс операционных систем
      \item Системные вызовы -- часть POSIX, но не все
      \item Например, POSIX описывает как должна быть устроена файловая система
    \end{itemize}
    Иными словами, POSIX -- это стандарт написания операционных систем. Windows -- не POSIX-совместимая система.
    
    \subsection{libc}
    \begin{itemize}
      \item Стандартная библиотека C
      \item Реализует системные вызовы в виде функций C
      \item Ещё куча всяких полезных функций :)
      \item Много реализаций, glibc одна из самых больших
    \end{itemize}
    POSIX определяет, как устроены системные вызовы в виде функций языка C.
    
    \subsubsection{Пример}
\begin{lstlisting}[style=cpp]
int res = read(0, &buf, 1024);
if (res < 0) {
  char* err = strerror(errno);
  // ...
}
\end{lstlisting}
Функция \lsin{read} возвращает $-1$, если считать не получилось, в противном случае -- количество записанных байт. \newline
\lsin{errno} -- это глобальная переменная (внутри одного потока), в которой хранится последняя ошибка.\newline
Вернуть массив из функции сложно (о причинах будет рассказано в следующих лекциях), поэтому обычно мы просим не вернуть результат, а записать его по некоторому адресу в памяти.

    \subsection{Файловые дескрипторы}
    \begin{itemize}
      \item ``Everything is a file!''
      \item Каждый файл имеет своё имя (или \textit{путь})
      \item Преобразовывать имя файла на каждый сисколл дорого
      \item Сначала нужно получить файловый дескриптор (например, через сисколл \lsin{open})
      \item Все остальные операции без использования пути
    \end{itemize}
    Файловый дескриптор -- это число. Например, $0$ -- это \lsin{stdin}, $1$ -- это \lsin{stdout}, $2$ -- \lsin{stderr}.

    \section{Представление данных в компьютере}
  \subsection{Беззнаковые типы}
  \begin{itemize}
    \item Представляют из себя $N$-битные положительные числа на отрезке $[0, 2^N - 1]$
    \item Переполнение точно определено стандартом C (как сложение в $\mathbb{Z}_{2^N}$
    \item $1111+0001=10000=0$
  \end{itemize}
    
  \subsubsection{Endianess}
    \begin{itemize}
      \item Если $N = 64$, то $64 / 8 = 8$ байт нужно, чтобы представить число в памяти
      \item Если $N = 32$, то $32 / 8 = 4$ байта
      \item В какой последовательности хранить биты?
    \end{itemize}
    
    Есть 2 типа endianess:
    \begin{itemize}
      \item Little-endian: первые байты хранят младшие биты числа
      \item Big-endian: первые байты хранят старшие биты
    \end{itemize}
    Сейчас более распространен Little-endian.\\    Традиционно Big-endian используется в передаче данных по сети. Также первые процессоры использовали big-endian. PowerPC тоже использует big-endian.\\
    На некоторых arm-процессорах есть инструкция, позволяющая менять endian "на лету". 
    
\begin{figure}[h!]
  \includegraphics[width=\linewidth]{/Users/user/Documents/endianess.png}
  \caption{Endianess}
  \label{fig:endianess}
\end{figure}

    
  \subsection{Выравнивание}
    \begin{itemize}
      \item Числа быстрее считываются процессором, если они лежат по адресам, кратным их размерам
      \item Например: \lsin{sizeof(int)} = 4 $\Rightarrow$ выравнивание по границе 4 байт
      \item \lsin{char} -- 1 байт
      \item \lsin{short} -- 2 байта
      \item \lsin{int} -- 4 байта
      \item \lsin{long long} -- 4 байта
    \end{itemize}
    
    Работа с выровненными данными происходит быстрее.\\
    Есть архитектуры, которые в принципе не позволяют читать по невыровненным адресам, например, arm. В процессорах Intel можно сделать так же.
    
  \subsubsection{Выравнивание структур}
    \begin{itemize}
      \item Члены структур располагаются рядом
      \item Но если им не хватает выравнивания, компилятор "добивает" структуру pad'ами
      \item Выравнивание структуры -- максимальное выравнивание среди всех выравниваний её членов
    \end{itemize}
    
  \subsection{Знаковые числа}
  \subsubsection{One's complemnt}
    \begin{itemize}
      \item $-A = BitwiseNot(A)$
      \item Диапазон: $[-2^{N - 1} + 1, 2^N - 1]$
    \end{itemize}
    Преимущество такого представления: естественным образом реализуется сложение чисел. Однако есть проблема.
    \begin{itemize}
      \item $-1 = 1110$
      \item $+1 = 0001$
      \item $1110 + 0001 = 1111 = -0$
    \end{itemize}
    Получается 2 представления нуля. Это порождает еще проблемы:
    \begin{itemize}
      \item $-1 = 1110$
      \item $+2 = 0010$
      \item $1110 + 0010 = \textcolor{red}{1}0000 = 0$
      \item Упс...
    \end{itemize}
  
  \subsubsection*{One's complement: end-around-carry}
    \begin{itemize}
      \item Бит переноса отправляется назад, чтобы всё исправить
      \item $1110 + 0010 = \textcolor{red}{1}0000 = 0 + \textcolor{red}{1} = 1$
    \end{itemize}
  
  \subsubsection*{One's complement: недостатки}
    \begin{itemize}
      \item Два представления для $0$: $0000 = +0$ и $1111 = -0$
      \item End-around-carry
      \item Зато сложение и вычитания одинаковое для знаковых и беззнаковых чисел (почти)!
    \end{itemize}
  
  \subsubsection{Two's complement}
    \begin{itemize}
      \item Определение отрицательных чисел: $A + (-A) = 0$
      \item Давайте каждому положительному число сопоставим отрицательное
      \item $-A = BitwiseNot(A) + 1$
      \item Одно представление нуля: $-0 = BitwiseNot(A) + 1 = 1111 + 1 = 0000 = +0$
      \item Диапазон чуть больше, чем у one's complement: $[-2^N, 2^N - 1]$
      \item Используется в современных процессорах
    \end{itemize}
    
    Теперь мы можем складывать знаковые и беззнаковые числа абсолютно одинаково.
  
  \subsubsection*{Two's complement: недостатки}
    \begin{itemize}
      \item Операции сравнения теперь сложные
      \item Умножение требует sign extension: $0010 = 00000010, 1000 = 11111000$
      \item ``Перекос'' диапазона представимых чисел
      \item \lsin{abs(INT_MIN)} = ???
    \end{itemize}
  
  \subsection{Действительные числа}
  \subsubsection{Числа с фиксированной точкой}
    \begin{itemize}
      \item $N$ бит на целую часть, $M$ бит на дробную
      \item Всегда одинаковая точность
      \item Операции легко реализуются
    \end{itemize}
  
  \subsubsection{Числа с плавающей точкой}
  Раньше процессоры имели отдельную плату для операций с числами с плавающей точкой.
    \begin{itemize}
      \item IEEE 754
      \item Стандарт 1985 года
    \end{itemize}
  
    Числа с плавающей точкой представлены 3 частями:
    \begin{itemize}
      \item Представление: $(-1)^S \times M \times 2^E$
      \item $S$ -- бит знака, $M$ -- мантисса, $E$ - экспонента
      \item \lsin{float} (single): $|S| = 1$, $|M| = 23$, $|E| = 8$
      \item \lsin{double}: $|S| = 1$, $|M| = 52$, $|E| = 11$
    \end{itemize}
  
  \subsubsection*{Нормализованные значения}
    \begin{itemize}
      \item $|E| \neq 0$ и $E \neq 2^{|E|} - 1$
      \item Экспонента хранится со смещением: $E_{real} = E - 2^{|E| - 1}$
      \item Мантисса имеет ``виртуальную 1'': $M_{real} = 1.mmmmmmm$
    \end{itemize}
    
  \subsubsection*{Денормализованные значения}
    \begin{itemize}
      \item $E = 0$
      \item $E_{real} = 1 - 2^{|E|} = 1$
      \item Это самые близкие к нулю числа и сам ноль ($0.0$ и $+0.0$)
    \end{itemize}
  
  \subsubsection*{Специальные значения}
    \begin{itemize}
      \item $E = 2^{|E| - 1}$
      \item Если $M = 0$, то число представляет собой бесконечное значение
      \item Если $M \neq 0$, то число -- NaN
      \begin{itemize}
        \item[$\circ$] Используются при операциях с неопределенным значением: например, $sqrt(X)$, $\log(X)$, $X < 0$
      \end{itemize}
    \end{itemize}
    
  \subsubsection*{Проблемы IEEE 754}
    \begin{itemize}
      \item При вычислениях накапливается ошибка
      \item Сложение и умножение неассоциативно
      \item Умножение недистрибутивно
      \item NaN $\neq$ NaN (???)
      \item $0.0$ и $+0.0$
    \end{itemize}
  
  Более подробно о числах с плавающей точкой можно прочитать \href{http://steve.hollasch.net/cgindex/coding/ieeefloat.html}{здесь}.
  
  \subsubsection{Decimals}
    \begin{itemize}
      \item Представляются в виде двух чисел: $N$ -- знаменатель, $M$ -- числитель
      \item Все операции реализуются через приведение к общему знаменателю
      \item $N$ и $M$ обычно используют длинную арифметику, поэтому в теории точность ограничена только оперативной памятью
      \item Используются в финансах
    \end{itemize}
  
  \subsection{Кодировки}
    \begin{itemize}
      \item Умеем оперировать числами, но как перевести числа в текст?
      \item Кодировки -- ``карты'', сопоставляющие наборы байт каким-то образом в символы
    \end{itemize}
  
  \subsubsection{Немного терминологии}
    \begin{itemize}
      \item Character -- что-то, что мы хотим представить
      \item Character set -- какое-то множество символов
      \item Coded character set (CCS) -- отображение символов в уникальные номера
      \item Code point -- уникальный номер какого-то символа
    \end{itemize}
    
  \subsubsection{ASCII}
    \begin{itemize}
      \item American Standard Code for Information Interchange, 1963 год
      \item 7-ми битная кодировка, то есть кодирует 128 различных символов
      \item Control characters: с $0$ по $31$ включительно, непечатные символы, мета-информация для терминалов
    \end{itemize}
  
  \subsubsection{Unicode}
    \begin{itemize}
      \item Codespace: $0$ до 0x10FFFF ($\sim$1.1 млн. code points)
      \item Code point'ы обозначаются как U+<число>
      \item $\aleph$ = U+2135
      \item $r$ = U+0072
      \item Unicode -- не кодировка: он не определяет как набор байт трактовать как characters
    \end{itemize}
  
  \subsubsection{UTF-32}
    \begin{itemize}
      \item Использует всегда 32 бита (4 байта) для кодировки
      \item Используется во внутреннем представлении строк в некоторых языках программирования (например, Python)
      \item Позволяет обращаться к произвольному code point'у строки за $\mathcal{O}(1)$
      \item BOM определяет little vs big-endian
    \end{itemize}
    Проблема: используется много места. Например, если мы пишем текст на английском, то под каждый символ будет выделено 4 байта, а можно было бы обойтись 1 (ASCII).
  
  \subsubsection{UTF-8}
    \begin{itemize}
      \item Unicode Transformation Format
      \item Определяет способ как будут преобразовываться code point'ы
      \item Переменная длина: от 1 байта (ASCII) до 4 байт
    \end{itemize}
    
    \begin{figure}[h!]
  \includegraphics[width=\linewidth]{/Users/user/Documents/utf8.png}
  \caption{UTF-8}
  \label{fig:utf8}
\end{figure}
  
  \subsubsection*{UTF-8 overlong encoding}
    \begin{itemize}
      \item $00100000$ = U+0020
      \item $11000000 \space 10100000$ = U+0020!
      \item overlong form или overlong encoding
      \item С точки зрения стандарта является некорректным представлением
    \end{itemize}
    \section{Файлы}
  \subsection{Файлы и директории}
    Файл -- это сущность, которая содержит данные и имеет имя.\\
    В Unix: Everything is a file!
  
  \subsubsection{Имя файла}
    \begin{itemize}
      \item Не более PATH\_MAX символов: 4 Кб на современных ОС, 256 байт для portability
      \item PATH\_MAX включает \textbackslash 0 в конце
      \item Разделитель пути -- /
      \item Части пути не более 255 символов каждая
    \end{itemize}
    
  \subsubsection{Имя директории}
    \begin{itemize}
      \item Абсолютный путь: начинается с корня (например, /Users/carzil/mipt)
      \item Относительный путь: вычисляется от текущей директории (например, carzil/mipt)
      \item . -- текущая директория (./carzil/mipt = carzil/mipt и ./carzil/./././mipt = ./carzil/mipt)
      \item .. -- директорий выше (/Users/carzil/mipt/.. = /Users/carzil)
    \end{itemize}
    
  \subsection{Файловые системы}
      \begin{itemize}
        \item Структура данных для организации хранения информации
        \item Метаданные -- информация о файле: дата последнего изменения, права доступа, создатель и так далее
        \item Работают поверх хранилища (HDD, SSD, NVMe)
        \item Хранилище традиционно разбивается на \textit{блоки}
        \item Размер блоков обычно 512 байт или 4Кб
      \end{itemize}
  
  \subsubsection{ext2}
    \begin{itemize}
      \item Linux, 1993 год
      \item inode -- физическое представление файла на диске: заголовок с метаинформацией + информация где он хранится
      \item Директории тоже хранятся в inode, так как директория -- файл!
    \end{itemize}

\begin{figure}[H]
  \centering
  \includegraphics[width=0.8\linewidth]{/Users/user/Downloads/ext2_inode.png}
  \caption{inode}
  \label{fig:inode}
\end{figure}  
  
  \subsubsection{ext4}
    \begin{itemize}
      \item 2006 год
      \item Де-факто стандартная файловая система для Linux
      \item Журналируемая
      \item Для больших директорий используется HTree
    \end{itemize}
  
    Во времена ext2 была проблема: считалось, что жесткие диски живут долго и работают безотказно, однако на самом деле это было не так. При записи файла, например, могла произойти ошибка на жестком диске, и из-за этого ext2 ломалась. По этой причине в ext3 сделали журнал. Журнал -- это область на диске, в которую записываются логи всех операций, которые выполняются с данными. В ext3 в журнал записываются блоки, которые мы меняем на диске. Если в процессе изменения структуры данных файловой системы случился сбой диска, то в логи не будет записан маркер конца транзакции изменения файловой системы. И когда мы будем восстанавливать файловую систему по журналу, мы просто применим все записанные транзакции и получим какое-то её стабильное состояние. Также у нас есть возможность обратить изменения на диске. Это нам гарантирует, что если посреди операции возникла ошибка, мы не сломаем все данные, и сможем всё восстановить. Журналируемость можно отключить, благодаря этому файловая система будет работать быстрее. \\
    Еще одно отличие ext2 и ext3: есть директории, в которых находятся очень много файлов. Из-за этого поиск в таких директориях работает долго: нужно пробежаться по всем записям в inode и сравнить названия файлов с тем, что мы ищем. Поэтому в ext3 (и в ext4 соответственно) была создана структура данных HTree, которая хранит кэши файловых имен. Благодаря этому можно за амортизированное $\mathcal{O}(1)$ искать файлы в директориях. Использование HTree тоже можно отключить.
    
  \subsubsection{Другие файловые системы}
    \begin{itemize}
      \item FAT32
      \item NTFS (проприетарная, используется в Windows)
      \item ReiserFS (оптимизирует работу с большим количеством маленьких файлов)
      \item ZPS (может работать с несколькими устройствами)
    \end{itemize}
  
  \subsubsection{sysfs и procfs}
    \begin{itemize}
      \item ``Метафайловые системы''
      \item Не имеют никаких данных на диске, возвращают информацию напрямую из ядра Linux
      \item Часто используются, чтобы не добавлять новые сисколлы
    \end{itemize}

  \subsubsection{FUSE}
    \begin{itemize}
      \item Код файловой системы обычно расположен в ядре -- это неудобно
      \item FUSE = file system in userspace
    \end{itemize}
    Когда мы хотим поменять файл, мы сообщаем об этом операционной системе, а та, видя, что мы обращаемся к FUSE-файловой системе, передаст этот запрос процессу файловой системы.\\
    Пример использования: можно работать с файлами на удаленной машине, как будто они находятся локально на нашем компьютере.  В таком случае все изменения файловой системы будут пересылаться по сети. Это можно сделать при помощи SSHFS.
  
  \subsection{Файловые дескрипторы}
    \begin{itemize}
      \item Преобразование имени файла в inode -- очень дорогая операция, которая может требовать много обращений к диску
      \item Этот процесс ``кэшируют'' с помощью файловых дескрипторов
      \item Файловый дескриптор -- число больше 0
      \item Новый файловый дескриптор будет минимальным доступным числом
    \end{itemize}
  Благодаря файловым дескрипторам можно только один раз делать преобразование имени файла в inode (делая сисколл open). При этом привязка произойдет именно к inode, а не к имени файла.
    \begin{itemize}
      \item За каждым файловым дескриптором скрывается \href{https://elixir.bootlin.com/linux/v5.19.12/source/include/linux/fs.h#L925}{специальная структура} в ядре
      \item Указатель на inode, позиция в файле, флаги (чтения/запись/блокирование), различные локи и так далее
    \end{itemize}
  
    \subsubsection{Работа с данными файла}
\begin{cminted}
#include <unistd.h>

int open(const char *pathname, int flags, mode_t mode);
ssize_t read(int fd, void *buf, size_t count);
ssize_t write(int fd, const void *buf, size_t count);
int close(int fd);
\end{cminted}
    
    \subsubsection{Работа с метаданными файла}
\begin{cminted}
#include <sys/stat.h>

int stat(const char* path, struct stat* buf);
int fstat(int fd, struct stat *statbuf);
int lstat(const char* pathname, struct stat* statbuf);

struct stat {
  dev_t     st_dev;
  ino_t     st_ino;
  mode_t    st_mode;
  nlink_t   st_nlink;
  uid_t     st_uid;
  gid_t     st_gid;
  dev_t     st_rdev;
  off_t     st_size;
  blksize_t st_blksize;
  blkcnt_t  st_blocks;
  struct timespec st_atime/st_mtime/st_ctime;
};
\end{cminted}
    
    \subsubsection{Права доступа}
      \begin{itemize}
        \item rwx = \textbf{R}ead/\textbf{W}rite/e\textbf{X}ecute
        \item 9 бит, 3 группы; права владельца, права группы и права для остальных
        \item Часто записываются как числа в восьмиричной системе счисления
        \item $777_8 = 111111111_2$ = \bmint{rwxrwxrwx}
        \item $664_8 = 110100100_2$ = \bmint{rw-r--r--}
      \end{itemize}
    
    \subsubsection{Права доступа для директорий}
      \begin{itemize}
        \item r -- листинг директории
        \item w -- создание файлов внутри директории
        \item x -- возможность перейти в директорию (cd),  а также доступ к файлам
      \end{itemize}
    
    \subsubsection{Регулярные файлы}
      \begin{itemize}
        \item \cmint{S_ISREG(stat.st_mode)}
        \item Обычные файлы с данными
      \end{itemize}
    
    \subsubsection{Директории}
      \begin{itemize}
        \item \cmint{S_ISDIR(stat.st_mode)}
        \item Специальный API для чтения, обычные read/write не работают
        \item Создание и удаление: mkdir/rmdir
      \end{itemize}
    
\begin{cminted}
#include <dirent.h>

struct dirent *readdir(DIR *dirp);

struct dirent {
  ino_t          d_ino;
  off_t          d_off;
  unsigned short d_reclen;
  unsigned char  d_type;
  char           d_name[256];
};

int closedir(DIR *dirp);
\end{cminted}
    
    \subsubsection{Символические ссылки}
      \begin{itemize}
        \item \cmint{S_ISLINK(stat.st_mode)}
        \item Аналог \cppmint{std::weak_ptr} для inode
        \item Могут быть dangling: то есть ссылаться на файл, которого нет
        \item Отдельный тип файла
        \item Путь, на который она ссылается, записан в блоках
      \end{itemize}
    
    \subsubsection{Жесткие ссылки}
      \begin{itemize}
        \item Аналог \cppmint{std::shared_ptr} для inode
        \item Только внутри одной файловой системы
        \item Если количество жестких ссылок стало равно 0, то inode становится свободной
        \item Не файл, а сущность файловой системы
      \end{itemize}
    
    \subsubsection{Символьные устройства (character device)}
      \begin{itemize}
        \item \cmint{S_ISCHR(stat.st_mode)}
        \item Устройства, из которых можно последовательно читать
        \item Клавиатура, звуковая карта, сетевая карта
        \item Такие файлы создаются драйверами ядра
      \end{itemize}
    
    \subsubsection{Блочные устройства (block device)}
      \begin{itemize}
        \item \cmint{S_ISBLC(stat.st_mode)}
        \item Разбиты на блоки одинакового размера
        \item Можно прочитать любой блок
        \item HDD, SSD, NAS
      \end{itemize}
    
\end{document}