\section{Сети, часть 1. Модель OSI}
  \subsection{Протоколы}
    \begin{itemize}
      \item Современные сети строятся на основе \textit{протоколов}
      \item Каждый протокол -- абстракция над другим протоколом (более низким) для решения какой-то проблемы
      \item Поэтому иногда говорят \textit{стек протоколов}
    \end{itemize}
  
  \subsection{Модель OSI}
  OSI разделяет современные протоколы на 7 уровней:
  \begin{itemize}
    \item Layer 1: Physical layer
    \item Layer 2: Data link layer
    \item Layer 3: Network layer
    \item Layer 4: Transport layer
    \item \textit{тут немного пропустим}
    \item Layer 7: Application layer
  \end{itemize}
  
  \subsubsection{Модель OSI: physical layer}
    \textit{Как данные будут переданы через физическую среду?}
    Протоколы:
    \begin{itemize}
      \item Bluetooth
      \item Ethernet physical layer: Ethernet over twisted pair, Fast Ethernet, Gigabit Ethernet, ...
      \item IEEE 802.11g/b/n
    \end{itemize}
  
  \subsubsection{Link layer}
    \textit{Как компьютеры могут общаться в локальной сети?}
    \begin{itemize}
      \item Обеспечивает обмен данными между узлами в одной сети LAN (local area network)
      \item Обычно на этом уровне протоколы оперируют пакетами (фреймами) (например, 1500 байт для Ether)
      \item На этом уровне появляется канальный адрес (link address), например, MAC-адрес
    \end{itemize}
    Есть 2 среды, в которых передаются данные. P2P, когда два участника соединены непосредственно между собой, и в их общение трудно вмешаться. А есть среды full-mesh.
  
  \subsubsection{Network layer}
    \textit{Как объединить локальные сети?}
    \begin{itemize}
      \item Обеспечивает связь между разными LAN
      \item На этом уровне появляется сетевой адрес (например, IP адрес)
      \item Появляется понятия \textit{маршрутизации}
      \item Основной протокол -- Internet Protocol (IP)
    \end{itemize}
  
  \subsubsection{Transport layer}
    \textit{Как обеспечить обмен данными между приложениями?}
    \begin{itemize}
      \item Используется для передачи данных между различными приложениями на узлах сети
      \item Используются \textit{порты}, чтобы разделять приложения на концах маршрутов
      \item Примеры: TCP, UDP, SCTP
    \end{itemize}
  
  \subsubsection{Application layer}
    \textit{Как каждое конкретное приложение обменивается данными?}
    \begin{itemize}
      \item Протоколы приложений (веб-браузеры, почтовые клиенты, игры)
      \item Один из самых известных -- HyperText Transfer Protocol (HTTP)
      \item Почтовые протоколы -- SMTP, POP3
      \item Secure Shell (SSH)
    \end{itemize}
  
  \subsection{Ethernet}
    \begin{itemize}
      \item Совокупность стандартов (IEEE 803.2), описывающие разные протоколы physical и link layer
      \item Передача между устройствами осуществляется с помощью фреймов
      \item Каждое устройство имеет свой MAC-адрес
      \item Ethernet 803.11 -- это point-to-point протокол, используются коммутаторы (или свитчи) для построения full-mesh сети
      \item Свитч имеет таблицу всех подключенных клиентов и форвардит фреймы в соответствии с ней
      \item \bmint{FF:FF:FF:FF:FF:FF} -- это специальный \textit{broadcast} адрес
    \end{itemize}
  
  \subsection{Internet Protocol}
    \begin{itemize}
      \item Протокол для объединения LAN
      \item Две версии: IPv4 vs IPv6
      \item IPv4 широко распространен, IPv6 только начинает появляться
    \end{itemize}
  
  \subsection{IP}
  \subsubsection{IP адреса}
    \begin{itemize}
      \item IPv4 адрес состоит из 32 бит (для IPv6 -- 128)
      \item Обычно записывается в виде 4 октетов (4 байт) через точку: 8.8.8.8
    \end{itemize}
  
  \subsubsection{Маска подсети}
    \begin{itemize}
      \item Множество IP-адресов с одинаковым \textit{префиксом}
      \item Маска подсети: \cmint{255.0.0.0}
      \item CIDR-нотация: \cmint{10.0.0.0/8}
      \item Private subnets: \cmint{10.0.0.0/8}, \cmint{127.0.0.1/8}, \cmint{192.168.0.0/16}
    \end{itemize}
  
  \subsubsection{IP: маршрутизация}
    \begin{itemize}
      \item Не все узлы сети связаны напрямую $\Rightarrow$ давайте передавать данные через другие узлы
      \item Передача между соседними узлами в IP-сети называется прыжком или хопом (hop)
      \item Каждый узел сети имеет \textit{таблицу маршрутизации}
      \item Эта таблица содержит маску подсетей и \textit{gateway} -- узел, куда нужно переслать данные
    \end{itemize}
  
  \subsubsection{ARP}
    \begin{itemize}
      \item Address Resolution Protocol
      \item Определяет MAC-адрес по IP
      \item Рассылается broadcast пакет `Who has 10.0.0.2? Tell 10.0.0.1'
      \item Все хосты проверяют свой IP и если он совпадает, отсылают `10.0.0.2 is at XX:XX:XX:XX:XX:XX'
      \item ARP spoofing
    \end{itemize}
  
  \subsubsection{IP: маршрутизация}
    \begin{itemize}
      \item AS -- это система IP-сетей и маршрутизаторов, управляемых одним или несколькими операторами, имеющими единую политику маршрутизации с Интернетом (\copyright Wiki)
      \item Блоки подсетей выдаются автономным сетям
      \item Точка обмена трафиком -- точки обмена трафиком между разными AS
      \item Одна из самых крупных в Европе и России -- MASK-IX
    \end{itemize}
  
  \subsubsection{IPv4: устройство пакета}
\begin{figure}[H]
  \centering
  \includegraphics[width=1\linewidth]{/Users/user/Downloads/ipv4_header.png}
  \caption{IPv4 Header format}
  \label{fig:ipv4_format}
\end{figure}
  
  \subsubsection{IP: TTL}
    \begin{itemize}
      \item TTL -- time to live
      \item Байт, который описывает максимальное количество прыжков в сети
      \item Если очередной хост уменьшил TTL до нуля, то пакет просто дропается, а отправителю посылается специальное сообщение по протоколу ICMP (TTL exceeded)
      \item На основе этого поведения работает trecroute/tracepath
    \end{itemize}
  
  \subsubsection{IP: проблемы}
    \begin{itemize}
      \item Не гарантирует доставку данных (packet loss)
      \item Не гарантирует порядок доставки (packet reordering)
      \item Не гарантирует, что пакет будет отправлен лишь один раз (packet duplication)
      \item Непонятно как реализовывать multitenancy IP протокола – нельзя всем приложениям рассылать все IP-пакеты
    \end{itemize}
  
  \subsection{BGP}
    \begin{itemize}
      \item Как операторам AS обновлять маршруты?
      \item Border gateway protocol 
      \item Устанавливается между BGP-роутерами соседних AS
      \item Каждая AS анонсирует свои префиксы (подсети)
      \item Изменения распространяются по всему интернету
      \item Выбирается самый `короткий' маршрут
      \item Таблица маршрутизации Интернета очень большая: на текущий момент 900k+ префиксов
    \end{itemize}
  
  \subsection{TCP}
    \begin{itemize}
      \item Вводит понятие \textit{порта} приложения -- адрес получателя на IP-узле
      \item Обеспечивает надежную доставку данных (reliable delivery)
      \item Обеспечивает порядок доставки и дедупликацию данных
      \item Connection-orinted -- приложения должны установить полнодуплексное \textit{соединение}
      \item Data stream -- данные передаются не отдельными пакетами, а непрерывным потоком
      \item Также обеспечивает congestion control и flow control
    \end{itemize}
  
  \subsubsection{TCP: устройство пакета}
\begin{figure}[H]
  \centering
  \includegraphics[width=1\linewidth]{/Users/user/Downloads/tcp_header.png}
  \caption{TCP segment format}
  \label{fig:tcp_format}
\end{figure}
  
  \subsubsection{TCP: 3-way handshake}
    Хендшейк -- механим установки соединения. Выполняется в три этапа:
    \begin{enumerate}
      \item Инициатор соединения (клиент) посылает пакет с флагом SYN серверу
      \item Сервер посылает пакет с флагами SYN и ACK клиенту, а также sequence number, с которого будут нумероваться все остальные байты
      \item Клиент посылает ACK
    \end{enumerate}
  
  \subsubsection{TCP}
    \begin{itemize}
      \item Transmission Control Protocol
      \item Отправитель посылает пронумерованные пакеты, что позволяет реконструировать правильную последовательность при переупорядочивании 
      \item Принимающий в ответ отправляет `одобрения': `принято все до номера N'
      \item Если пакет теряется, то acknowledgment (ACK) долго не приходит, отправитель перепосылает пакет
      \item Если пришел повторный ACK, то отправитель тоже перепосылает пакет
    \end{itemize}
  
  \subsection{UDP}
    \begin{itemize}
      \item User Datagram Protocol
      \item Не даёт никаких гарантий
      \item Пересылка осуществляется через пакеты (UDP datagrams)
      \item Используется там, где неважна последовательность пакетов (например, онлайн игры или торренты)
      \item Или там, где потеря/переупорядочивание/дублирование не сильно сказывается (например, звонки)
    \end{itemize}
  
  \subsection{DNS}
    \begin{itemize}
      \item Позволяет преобразовывать человекочитаемые доменные имена в IP-адреса
      \item Работает поверх UDP, стандартный порт – 53
      \item Корневые сервера обслуживают все запросы DNS в интернете
      \item Результат кэшируется и имеет время жизни, заданное в ответе DNS (TTL)
      \item Корневые сервера спускаются к более мелким DNS серверам (например, к .ru)
      \item Более мелкие могут спускаться дальше итд – recursive resolving
    \end{itemize}
  
  \subsubsection{DNS: виды записей}
    \begin{itemize}
      \item \textbf{A} запись: имя -> IPv4 адрес
      \item \textbf{AAAA} запись: имя -> IPv6 адрес
      \item \textbf{NS} запись: имя -> авторитетный DNS-сервер
      \item \textbf{CNAME} запись: имя -> имя
    \end{itemize}
  
  \subsection{HTTP}
    \begin{itemize}
      \item HyperText Transport Protocol
      \item L7 протокол
      \item Запрос-ответ: клиент отправляет запросы, сервер возвращает ответами
      \item Человеко-читаемый
      \item Для перевода строк служит \textbackslash r\textbackslash n
    \end{itemize}
  
  \subsubsection{HTTP: запрос}
    \begin{itemize}
      \item Метод: GET, POST, DELETE, PUT, OPTIONS, HEAD
      \item Uniform Resource Identifier (URI) – путь запроса
      \item Версия HTTP
      \item Заголовки
      \item Тело (опционально)
    \end{itemize}
    
    \begin{bminted}
POST /cgi-bin/process.cgi HTTP/1.1
User-Agent: Mozilla/4.0 (compatible; MSIE5.01; Windows NT)
Host: www.tutorialspoint.com
Content-Type: application/x-www-form-urlencoded
Content-Length: length
Accept-Language: en-us
Accept-Encoding: gzip, deflate
Connection: Keep-Alive

licenseID=string&content=string&/paramsXML=string  
    \end{bminted}
  
  \subsubsection{HTTP: ответ}
    \begin{itemize}
      \item Версия HTTP
      \item Код (статус) ответа
      \item Расшифровка ответа (reason)
    \end{itemize}
    
        \begin{bminted}
HTTP/1.1 404 Not Found
Date: Sun, 18 Oct 2012 10:36:20 GMT
Server: Apache/2.2.14 (Win32)
Content-Length: 230
Connection: Closed
Content-Type: text/html; charset=iso-8859-1
    \end{bminted}
  
  \subsubsection{HTTP: статус коды}
    \begin{itemize}
      \item \textbf{1xx} – информационные
      \item \textbf{2хх} – успешные коды (например, 200 OK)
      \item \textbf{3xx} – для перенаправлений пользователей (редиректы)
      \item \textbf{4xx} – ошибка клиента (неправильный адрес, некорретный запрос)
      \item \textbf{5xx} – ошибка сервера (внутреняя ошибка, сервис временно недоступен)
    \end{itemize}

