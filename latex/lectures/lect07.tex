\section{Сигналы}
  \begin{itemize}
    \item Асинхронное событие, которое может произойти после любой инструкции
    \item Используются для уведомления процессов о каких-то событиях
    \item Сигнал можно либо обрабатывать, либо игнорировать
    \item Однако \cmint{SIGKILL} и \cmint{SIGSTOP} нельзя обработать или проигнорировать
  \end{itemize}

  \subsection{Сигналы: примеры}
    \begin{itemize}
      \item Нажатие \bmint{^C} (Ctrl+C) в терминале генерирует \bmint{SIGINT}
% looking weird
      \item Нажатие \bmint{^}\cmint{\} (Ctrl+Backslash) в терминале генерирует \bmint{SIGQUIT}
      \item Запись в пайп только с write-концом генерирует \bmint{SIGPIPE}
      \item вызов \bmint{abort()} приводит к \bmint{SIGABORT}
      \item Обращение к несуществующей памяти генерирует \bmint{SIGSEGV}
    \end{itemize}
  
  \subsection{Доставка сигналов}
    \begin{itemize}
      \item Сигналы могут быть доставлены от ядра (например, \bmint{SIGKILL}, \bmint{SIGPIPE})
      \item Либо от другого процесса (\bmint{SIGHUP}, \bmint{SIGINT}, \bmint{SIGUSR})
      \item Или процесс может послать сигнал сам себе (\bmint{SIGABRT})
      \item Сигналы могут быть доставлены в \textit{любой момент} выполнения программы
    \end{itemize}
  
  \subsection{Signal safety}
    \begin{itemize}
      \item Во время обработки сигналы процессы могут быть в критической секции
      \item Поэтому в обработчиках сигналов нельзя использовать, например, \cmint{printf}
      \item Можно использовать только async-signal-safe функции
      \item \bmint{man 7 signal-safety}
    \end{itemize}
  
  \subsection{Обработка сигналов}
    \begin{cminted}
void signal_handler(int sig) {
  // ...
}

int main() {
  signal(SIGINT, signal_handler);
  signal(SIGTERM, signal_handler);
  signal(SIGSEGV, SIG_IGN);
  signal(SIGABRT, SIG_DFL);
}      
    \end{cminted}
    
  \subsection{Посылка сигналов}
    \begin{cminted}
#include <signal.h>

int raise(int sig);
int kill(pid_t pid, int sig);
    \end{cminted}
Название \cmint{kill} существует по историческим причинам: раньше был только один сигнал -- \bmint{SIGKILL}.
  
  \subsubsection{Посылка сигналов: аргументы kill}
    \begin{itemize}
      \item Если \cmint{pid == 0}, то сигнал будет доставлен текущей группе процессов
      \item Если \cmint{pid > 0}, то сигнал будет доставлен процессу pid
      \item Если \cmint{pid == -1}, то сигнал будет всем процессам, которым текущий процесс может его отправить
      \item Если \cmint{pid < -1}, то сигнал будет доставлен группе процессов \cmint{-pid}
      \item Если \cmint{sig == 0}, то сигнал не будет никому отправлен, а будет только осуществлена проверка ошибок (dry run)
      \item Возврат из kill не гарантирует, что сигнал обработался в получателе(-ях)!
    \end{itemize}
  
  \subsection{Доставка сигналов: маски сигналов}
    \begin{itemize}
      \item Маска сигналов — bitset всех сигналов
      \item У процесса есть две маски сигналов: \textit{pending} и \textit{blocked}
      \item pending — это те сигналы, которые должны быть доставлены, но ещё не успели
      \item Из этого следует, что если несколько раз отправить сигнал в один процесс, то он может быть обработан лишь единожды
      \item blocked — это те сигналы, которые процесс блокирует (блокировать и игнорировать — разные вещи)
      \item Если сигнал заблокирован, это значит, что он не будет доставлен вообще, если проигнорирован — то у него просто пустой обработчик
    \end{itemize}
    
    \begin{cminted}
#include <signal.h>

int sigemptyset(sigset_t* set);
int sigfillset(sigset_t* set);
int sigaddset(sigset_t* set, int signum);
int sigdelset(sigset_t* set, int signum);
int sigismember(const sigset_t* set, int signum);
    \end{cminted}

  \subsubsection{Доставка сигналов: \cmint{sigprocmask}}
    \begin{itemize}
      \item \cmint{int sigprocmask(int h, sigset_t *set, sigset_t *oset);}
      \item SIG\_SETMASK — установить маску заблокированных сигналов
      \item SIG\_BLOCK — добавить сигналы \cmint{set} в заблокированные
      \item SIG\_UNBLOCK — удалить сигналы \cmint{set} из заблокированных
      \item Если \cmint{oset != NULL}, то туда будет записана предыдущая маска
    \end{itemize}
  
  \subsection{Обработка сигналов: \cmint{sigaction}}
    \begin{cminted}
#include <signal.h>

int sigaction(int signum, const struct sigaction* act,
              struct sigaction* oldact);

struct sigaction {
  void     (*sa_handler)(int);
  void     (*sa_sigaction)(int, siginfo_t*, void*);
  sigset_t   sa_mask;
  int        sa_flags;
};  
    \end{cminted}

    \begin{itemize}
      \item Выставляет обычный обработчик \cmint{sa_handler}
      \item Или расширенный: \cmint{sa_sigaction}
      \item При выполнении сигнала \cmint{signum} в заблокированные сигналы добавятся сигналы из \cmint{sa_mask}, а также сам сигнал
      \item \cmint{sa_flags} — флаги, меняющие поведение обработки
      \item Чтобы использовать \cmint{sa_sigaction}, нужно выставить \cmint{SA_SIGINFO}
      \item Если выставить \cmint{SA_RESETHAND}, то обработчик сигнала будет сброшен на дефолтный после выполнения
      \item Если выставить \cmint{SA_NODEFER}, то если сигнал не был в \cmint{sa_mask}, обработчик может быть прерван самим собой
    \end{itemize}
  
  \subsubsection{Обработка сигналов: \cmint{siginfo_t}}
    \begin{cminted}
#include <signal.h>

siginfo_t {
  int      si_signo;     int      si_overrun;
  int      si_errno;     int      si_timerid;
  int      si_code;      void*    si_addr;
  int      si_trapno;    long     si_band;
  pid_t    si_pid;       int      si_fd;
  uid_t    si_uid;       short    si_addr_lsb;
  int      si_status;    void*    si_lower;
  clock_t  si_utime;     void*    si_upper;
  clock_t  si_stime;     int      si_pkey;
  sigval_t si_value;     void*    si_call_addr;
  int      si_int;       int      si_syscall;
  void*    si_ptr;       unsigned int si_arch;
}  
    \end{cminted}
  
  \subsection{SIGCHLD}
    \begin{itemize}
      \item Еще один способ уведомлять процессы о завершении дочерних
      \item \cmint{si_status} хранит информацию о том, как завершился процесс (\cmint{CLD_KILLED}, \cmint{CLD_STOPEED}, etc)
      \item \cmint{si_pid} хранит PID дочернего процесса
      \item Если выставить \cmint{SIG_IGN}, то зомби не будут появляться
    \end{itemize}
  
  \subsection{Доставка сигналов во время системных вызовов}
    \begin{itemize}
      \item Если сигнал прервал выполнение блокирующего сисколла, то есть два поведения
      \item Если использован \cmint{sigaction} и в \cmint{sa_flags} есть \cmint{SA_RESTART}, то после того, как обработчик завершится, сисколл продолжит свою работу (syscall restarting)
      \item Если не указан, то сисколл вернёт ошибку и \cmint{errno == EINTR}
    \end{itemize}

  \subsection{Ожидание сигналов: \cmint{pause}, \cmint{sigsuspend} и \cmint{sigwaitinfo}}
    \begin{itemize}
      \item \cmint{int pause(void)}
      \item Блокируется до первой доставки сигналов (которые не заблокированы)
      \item \cmint{int sigsuspend(const sigset_t* mask)}
      \item Атомарно заменяет маску заблокированных сигналов на mask и ждёт первой доставки сигналов
      \item \cmint{int sigwaitinfo(const sigset_t* restrict set, siginfo_t* restrict info)}
      \item Требует вызова \todo{что-то}
    \end{itemize}
  
  \subsection{Как устроены сигналы?}
    \begin{itemize}
      \item Ядро проверяет, нужно ли доставить сигнал в текущий процесс (при выходе из сисколла, или в S-состоянии, или по таймеру)
      \item Конструируется специальный отдельный стек (\cmint{sigaltstack})
      \item В начало стека кладётся фрейм, который содержит информацию о прерванной инструкции (\cmint{siginfo_t->ucontext})
      \item Ядро «прыгает» в обработчик события, выполняя его на этом стеке
      \item Обработчик завершается и вызывает \cmint{sigreturn}
      \item Ядро возвращается в предыдущий стек, инструкцию и так далее
    \end{itemize}
  
  \subsection{Наследование сигналов: fork и execve}
    \begin{itemize}
      \item \cmint{fork} сохраняет маску сигналов и назначенные обработчики
      \item \cmint{execve} сохраняет \textbf{\textit{только}} маску заблокированных сигналов
    \end{itemize}
  
  \subsection{Почему сигналы -- это плохо?}
    \begin{itemize}
      \item Почти невозможно обработать сигналы без race condition'ов
      \item Обработчики сигналов могут вызываться во время работы других обработчиков
      \item Посылка нескольких сигналов может привести к посылке только одного
      \item Не используйте сигналы для IPC (interprocess communication)!
    \end{itemize}
  
  Если в процессе несколько тредов, какой из них получит сигнал?
\begin{quote}\itshape
[...] A process-directed signal may be delivered to any one of the
threads that does not currently have the signal blocked.  If more
than one of the threads has the signal unblocked, then the kernel
chooses an arbitrary thread to which to deliver the signal [...]
\end{quote}

  \subsection{Почему лучше всегда использовать sigaction?}
    \begin{itemize}
      \item Война поведений: BSD vs System-V
      \item В отличие от BSD, в System-V обработчик сигнала выполняется единожды, после чего сбрасывается на обработчик по умолчанию
      \item В BSD обработчик сигнала не будет вызван, если в это время уже выполняется обработчик того же самого сигнала
      \item В BSD используется syscall restarting, в System-V — нет
      \item BSD (\cmint{_BSD_SOURCE} или \cmint{-std=c99}): \cmint{SA_RESTART}
      \item System-V (\cmint{_GNU_SOURCE} или \cmint{-std=gnu99}): \cmint{SA_NODEFER|SA_RESETHAND}
      \item Всегда лучше использовать \cmint{sigaction} для однозначности поведения программы!
    \end{itemize}
  
  \subsection{Real-time signals}
    \begin{itemize}
      \item При посылке сигналов учитывается их количество и порядок
      \item Отделены от обычных: начинаются с \cmint{SIGRTMIN}, заканчиваются \cmint{SIGRTMAX}
      \item Вместе с сигналом посылается специальная метаинформация (правда, только число), которую можно как-то использовать
      \item Получить эту дополнительную информацию можно через \cmint{siginfo_t->si_value}
    \end{itemize}
  
    \begin{cminted}
#include <signal.h>

union sigval {
  int    sival_int;
  void*  sival_ptr;
};

int sigqueue(pid_t pid, int signum, const union sigval value);  
    \end{cminted}

  \subsection{Обработка сигналов с помощью пайпов}
    \begin{itemize}
      \item Иногда не хочется возиться с атомарными счётчиками или хочется выполнить какие-то нетривиальные действия в обработчике
      \item Или в обработчике нет какого-то нужного контекста (экземпляра класса итд)
      \item Помогает трюк с пайпами
      \item В обработчике будем писать номер сигнала (или какую-то другую информацию) в пайп
      \item В основной программе будем делать read на другой конец пайпа
      \item \textbf{\textit{Важно}}: write-конец пайпа должен быть с флагом \cmint{O_NONBLOCKING}, иначе возможен дедлок
    \end{itemize}
  
  \subsection{Использование сигналов}
  \subsubsection{nginx}
    \begin{itemize}
      \item Если поменялся конфиг nginx, то как его подхватить заново?
      \item Постоянный траффик от клиентов $\Rightarrow$ перезапуск невозможен
      \item Сигналы приходят на помощь!
      \item \cmint{SIGHUP} сигнализирует nginx о том, что конфиг поменялся и его нужно перечитать и применить
      \item \cmint{SIGTERM} сигнализирует nginx о том, что нужно остановить обработку запросов как можно скорее и завершиться
      \item \cmint{SIGUSR1} используется для hot upgrade
    \end{itemize}
  
  \subsubsection{golang}
    \begin{itemize}
      \item Реализация preemptive multitasking
      \item Внутри Go реализованы лёгкие потоки -- горутины (coopeative multitasking)
      \item Иногда горутины могут зависать (если, например, много вычислений) и их нужно уметь принудительно вытеснять
      \item Отдельный тред \cmint{sysmon}, который следит за остальными тредами, исполняющими горутины
      \item Если какая-то горутина зависает больше, чем на 10 мс, посылается \cmint{SIGURG} и её выполнение прерывается
    \end{itemize}
