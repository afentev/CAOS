\section{Системные вызовы}
  \subsection{Зачем нужны системные вызовы?}
    Операционная система делает следующее:

    \begin{itemize}
      \item Управляет ресурсами устройства
      \item Строит и предоставляет абстракции для каких-то осмысленных действий с использованием этих ресурсов \begin{itemize}
        \item[$\circ$] Примеры: работа с файлами или процессами, ввод и вывод данных
      \end{itemize}
    \end{itemize}
    
    Системный вызов -- это способ попросить ядро выполнить для нас определённое действие
    \begin{itemize}
      \item Набор системных вызовов -- это интерфейс, который предоставляет нам операционная система
      \item Чем богаче этот интерфейс, тем больше у операционной системы возможностей
    \end{itemize}
  
  \subsection{Примеры системных вызовов POSIX}
    \begin{itemize}
      \item Управление процессами: \cmint{fork}, \cmint{waitpid}, \cmint{exec}, \cmint{exit}, \cmint{kill}
      \item Управление файлами: \cmint{open}, \cmint{close}, \cmint{read}, \cmint{write}, \cmint{lseek}, \cmint{stat}
      \item Управление каталогами и файловой системой: \cmint{mkdir}, \cmint{rmdir}, \cmint{link}, \cmint{unlink}, \cmint{mount}, \cmint{unmount}, \cmint{chdir}, \cmint{chmod}
      \item Управление памятью: \cmint{brk}, \cmint{mmap}, \cmint{munmap}
      \item Работа с сетью: \cmint{socket}, \cmint{bind}, \cmint{connect}, \cmint{listen}, \cmint{accept}
    \end{itemize}
    \textit{Примечание: одноимённые функции являются обёрткой над настоящими вызовами}
  
  \subsection{Безопасность}
    Операционная система также обеспечивает и безопасность:
    \begin{itemize}
      \item Процессы изолированы друг от друга (IPC реализовано средствами самой ОС)
      \item Пользовательские процессы не имеют прямого доступа к железу и работают в непривилегированном режиме CPU
      \item Операционная система имеет прямой доступ к железу и работает в привилегированном режиме CPU
      \item Системные вызовы позволяют пользовательским процессам выполнить нужный им код из ядра (конечно, если ОС не против)
    \end{itemize}
  
  \subsection{User Space \& Kernel Space}
  
\begin{figure}[H]
  \centering
  \includegraphics[width=1.0\linewidth]{/Users/user/Downloads/user_and_kernel_space.jpg}
  \caption{User \& Kernel space}
  \label{fig:spaces}
\end{figure}
  
  \subsection{Protection Rings}
    Описывают упомянутые выше уровни привилегий. Реализуются средствами CPU: в зависимости от уровня привилегий доступны разные наборы инструкций. Linux, macOS и Windows используют только два уровня
\begin{figure}[H]
  \centering
  \includegraphics[width=0.55\linewidth]{/Users/user/Downloads/protection_rings.png}
  \caption{Protection rings}
  \label{fig:rings}
\end{figure}
  
  \subsection{Как происходит системный вызов?}
    \textit{В \href{https://github.com/carzil/caos-2022/blob/master/lectures/10-syscalls-dynlibs/main.md}{оригинальной презентации} Сергея Горбунова есть очень понятные картинки, рекомендуется этот параграф читать там.}
    \subsubsection{Исходная позиция}
      Рассмотрим виртуальное адресное пространство нашего процесса, а также процессор, на котором он исполняется. Пусть в тексте программы происходит вызов какой-то библиотечной функции, тогда:

    \begin{itemize}
      \item Производим действия в соответствии с Calling Conventions
      \item Выполняем инструкцию \asmmint{call}
    \end{itemize}

    \subsubsection{Инструкция \asmmint{syscall}}
      Как мы уже знаем, взаимодействие с ОС выражается через системные вызовы, пусть мы дошли до одного из них. С точки зрения инструкций это будет выглядеть так:
    \begin{itemize}
      \item Кладём в регистры номер системного вызова и аргументы
      \item Выполняем инструкцию \asmmint{syscall}
    \end{itemize}

  \subsubsection{Обработка системного вызова}
      Что происходит дальше:

    \begin{itemize}
      \item Пользовательские страницы защищаются от чтения, записи и исполнения
      \item CPU переходит в привилегированный режим
      \item Пользовательский стек заменяется на стек ядра, сохраняется контекст
      \item Происходит поиск обработчика в заранее заполненной \bmint{System Calls Table}
    \end{itemize}
    
    Далее:

    \begin{itemize}
      \item Если обработчик найден, то выполняем его код, иначе сразу переходим к следующему шагу
      \item Выполняется инструкция \asmmint{sysret}, всё возвращается в состояние до выполнения \asmmint{syscall}
      \item Переходим к следующей инструкции
    \end{itemize}
      
    \subsubsection{Часто используемые системные вызовы}
      Существует класс системных вызовов, которые используются только для получения какой-то информации, причём и в User Space, и в Kernel Space результат выполнения будет одинаковым (например, \cmint{gettimeofday}). Для них существует оптимизация под названием \cmint{vDSO}
    
    \subsubsection{Virtual Dynamic Shared Object (vDSO)}
      Особая динамическая библиотека, автоматически отображаемая ядром в виртуальное адресное пространство каждого процесса. Содержит в себе реализации описанных выше системных вызовов. Исполнение происходит в User Space, не требует перехода в режим ядра, а значит увеличивает производительность
    
    \subsubsection{Финиш}
      Таким же образом выполняются и все остальные системные вызовы внутри вызванной нами функции. В конце мы возвращаемся к нашей программе и продолжаем её выполнение
  
  \subsection{Shared libraries}
    \subsubsection{Библиотеки}
      Мы можем создавать исполняемый код, который будут использовать другие программы. Для этого нам нужно создать библиотеку:
      \begin{itemize}
        \item Статическую (\bmint{Static}): \bmint{$ gcc -c -fPIC -o libcaos.a libcaos.c}
        \item Или разделяемую (\bmint{Shared}): \bmint{$ gcc -shared -fPIC -o libcaos.so libcaos.c}
      \end{itemize}
      Флаг \bmint{-fPIC} говорит о том, что полученный код будет позиционно независимым
    
    \subsubsection{Разделяемые библиотеки}
      Существует два способа использовать разделяемую библиотеку:
      \begin{itemize}
        \item \bmint{Dynamic Linking} -- динамическая линковка. При запуске программы будет произведён поиск зависимостей и дальнейшее разрешение неизвестных адресов
        \item \bmint{Dynamic Loading} -- разделяемые библиотеки можно загружать прямо во время работы программы с помощью \bmint{libdl}
      \end{itemize}
      Использование разделяемых библиотек позволяет уменьшить размер исполняемых файлов, однако требует наличия этих библиотек у конечного пользователя
      
    \subsubsection{Address Space Layout Randomization (ASLR)}
      Ряд эксплойтов опирается на то, что некоторые участки кода загружаются в одни и те же места. В целях безопасности существует механизм ASLR, который должен противостоять таким атакам путём рандомизации загрузки ключевых частей адресного пространства. В Linux текст программы, куча, разделяемые библиотеки и даже ядро (kASLR) загружаются со случайным смещением
