\section{Процессы}
    \begin{itemize}
      \item POSIX: ``A process is an abstraction thtat represents an executing program. Multiple processes execute independetly and have separate address spaces. Processes can ceate, interrupt, and terminate other processes, subject to secutiry restrictions''
      \item В самом ядре Linux нет понятия ``процесс'', вместо этого оно оперирует ``тасками''
      \item То, что в POSIX процесс, в Linux называется thread group
      \item Процессы объединяются в \textit{группы процессов}
      \item Группы процессов объединяются в \textit{сессии}
    \end{itemize}
  
  \subsection{Атрибуты процесса}
    \begin{itemize}
      \item Сохранённый контекст процессора (регистры)
      \item Виртуальная память (анонимные, private/shared, файловые)
      \item Файловые дескрипторы
      \item Current working directory (cwd)
      \item Текущий корень (\bmint{man 2 chroot})
      \item umask
      \item PID, PPID, TID, TGID, PGID, SID
      \item Resource limits
      \item Priority
      \item Capabilities
      \item Namespaces
    \end{itemize}
  
  \subsection{Состояния процессов}
    Runnable -- значит, что процесс прямо сейчас готов исполнять инструкции.
    Uninterruptible sleep -- состояние, пока процесс ждет ответа от операционной системы (например, выделяет память)
    Interruptible sleep -- состояние, когда процесс ``спит''. Например, мы вызвали \cmint{sleep}
    Zombie -- это особое состояние, про которое мы поговорим подробнее позже
    Stopped -- состояние, часто используемое, например, в дебаггерах.
\begin{figure}[H]
\centering
  \includegraphics[width=0.9\linewidth]{/Users/user/Downloads/task-states.png}
  \caption{Task states}
  \label{fig:task_states}
\end{figure}  
  
  \subsection{\mintinline{c}{fork}}
    \begin{itemize}
      \item Создать новый процесс можно только \textit{скопировав} текущий с помощью \cmint{pid_t fork()}
      \item \cmint{fork} выйдет в двух процессах одновременно, но в одном вернёт PID ребенка, а в другом -- 0.
      \item Ребенок будет полностью идентичен родителю, но файловые дескрипторы и адресное пространство будут \textit{скопированы}
      \item Для оптимизации потребления памяти используется copy-on-write подход для копирования памяти 
    \end{itemize}
\begin{cminted}
pid_t pid = fork();
if (pid < 0) {
  // error! can't create new process
} else if (pid == 0) {
  // We're in fork's child
  // getpid() returns current pid
} else {
  // We're in fork's parent and pid is the child's pid
}
\end{cminted}
Есть системный вызов \cmint{getpid}, возвращающий PID текущего процесса.

  \subsection{\cmint{execve}}
    \begin{itemize}
      \item Создавать копии недостаточно, нужно уметь запускать произвольные файлы
      \item Для этого используется системный вызов \cmint{execve}
      \item Он заменяет текущий процесс процессом, созданным из указанного файла
      \item Это называется заменой образа процесса: заменяются только части адресного пространства 
      \item Также в новом образе процесса останутся незакрытые файловые дескрипторы, не помеченные флагов \cmint{O_CLOEXEC}
    \end{itemize}

\begin{cminted}
extern char** environ;
int exec1(const char* path, const char* arg0, ...);
int execle(const char* path, const char* arg0, ...);
int execlp(const char* file, const char* arg0, ...);
int execv(const char* path, char* const argv[]);
int execve(const char* path, char* const argv[],
           char* const envp[]);  // system call!
int execvp(const char* file, char* const argv[]);
int fexecve(int fd, char* const argv[], char* const envp[]);
\end{cminted}

  \subsubsection{\cmint{execve}: сохраняемые атрибуты}
    \begin{itemize}
      \item В отличие от \cmint{fork} сохраняет меньше атрибутов
      \item Файловые дескрипторы (не помеченные флагом \cmint{O_CLOEXEC)}
      \item cwd и root
    \end{itemize}
  
\begin{cminted}
pid_t pid = fork();
if (pid == -1) {
  
} else if (pid == 0) {
  char* argv[] = {"ls", "-lah", NULL};
  char* envp[] = {"FOO=bar", "XYZ=abc", NULL};
  execve("/usr/bin/ls", argv, envp);
  // if you ended up here, something went wrong!
} else {

}
\end{cminted}

  \subsection{Атрибуты процесса: PID, PPID, TGID, ...}
    \begin{itemize}
      \item PID = process ID
      \item PPID = parent process ID
      \item PGID = process group ID
      \item SID = session ID
      \item \cmint{pid_t getpid()}, \cmint{pid_t, getpid()}, \cmint{pid_t getid()}
      \item \cmint{pid_t getpgid()}, \cmint{int setpgid(pid_t pid, pid_t pgid)}
      \item \cmint{pid_t setid()}, \cmint{pid_t getsid(pid_t pid)}
      \item \bmint{/proc/<pid>/status} ИЛИ В \bmint{/proc/<pid>/stat}
    \end{itemize}

  \subsection{Атрибуты владельца процесса}
    \begin{itemize}
      \item UID (user ID или real user ID) -- ID владельца процесса, \cmint{void setuid(uid_t)}
      \item EUID (effective user ID) используется для проверок доступа, \cmint{seteuid(uid_t)}
      \item SUID (saved user ID) используется, чтобы можно было временно пониизть привилегии
      \item FSUID (file system user ID) обычно совпадает с EUID, но может быть отдельно изменен через \cmint{int setfsuid(uid_t fsuid)}
      \item Непривилегированный процесс может выставлять EUID равный только в SUID, UID или опять в EUID
      \item Также есть понятие setuid/setgid флагов (sticky flags), в отличие от обычных файлов, EUID такого процесса будет выставлен как UID \textit{владельца файла}, а не \textit{текущий пользователь}
      \item Есть аналогичные GID, EGID, SGID, FSGID
    \end{itemize}
  
  \subsection{Работа с процессами}
  \subsubsection{exit}
    \begin{itemize}
      \item Завершает текущий процесс с определенным \textit{кодом возврата} (exit code)
      \item \cmint{exit} vs \cmint{_exit}
      \item Чтобы завершить текущую thread group можно воспользоваться \cmint{exit_group}
      \item exit закрывает все открытые файловые дескрипторы, освобождает выделенные страницы, etc
      \item Если у процесса были дети, то их родителем станет процесс с PID == 1
      \item После этого процесс становится \textit{зомби-процессом}
      \item Ядро не хранит огромную структуру для него, а только его PID и exit code
    \end{itemize}
  
  \subsubsection{wait}
    \begin{itemize}
      \item Дожидается, пока процесс будет остановлен
      \item Для этого использются системные вызовы семейства \cmint{wait*}
      \item Они дожидаются завершения процесса (конкретного или любого) и возвращают специальный \textit{exit status}
      \item Обычно exit status содержит то, что передали в \mintinline{c}{exit}
    \end{itemize}

\begin{cminted}
pid_t wait(int* stat_loc);
pid_t waitpid(pid_t pid, int* stat_loc, int options);
pid_t wait3(int* stat_loc, int options, struct rusage* rusage);
pid_t wait4(pid_t pid, int* stat_loc, int options,
            struct rusage* rusage);  // system call!
\end{cminted}

  \subsection{wait4}
    \begin{itemize}
      \item \cmint{pid < 1}: ждёт любой дочерний процесс в группе процессов \cmint{-pid}
      \item \cmint{pid == -1}: ждёт любой дочерний процесс
      \item \cmint{pid == 0}: ждёт любой дочерний процесс в текущей группе процессов
      \item \cmint{pid > 0}: ждёт конкретного ребенка
    \end{itemize}
  
  \subsubsection{Макросы для wait4}
\begin{cminted}
WIFEXITED(status)  // did the process terminate itself?
WEXITSTATUS(status)  // get exit status
WIFSIGNALED(status)  // did the process terminate by a signal?
WTEMSIG(status)  // get the signal
WCOREDUMP(status)  // did the process leave a coredump?
\end{cminted}

  \subsection{Лимит ресурсов процессов}
    \begin{itemize}
      \item Лимиты делятся на два типа: soft и hard
      \item Soft-лимит или текущий лимит -- по нему вычисляются проверки
      \item Hard-лимит -- максимальное значение soft-лимита
      \item CPU-время, если процесс его превысит ему ядро отошлет \textbf{SIGXCPI}, если превысит hard-лимит, то \textbf{SIGKILL}
      \item Размер записываемых файлов: write будет возвращать \textbf{EFBIG}
      \item Размер записываемых coredump-файлов
      \item На максимальный размер виртуальной памяти, выделенной процессу
      \item Количество одновременных процессов пользователя
      \item Количество файловых дескрипторов
    \end{itemize}

\begin{cminted}
#include <sys/time.h>
#include <sys/resource.h>

struct rlimit {
  rlim_t rlim_cur;  /* Soft limit */
  rlim_t rlim_max;  /* Hard limit (ceiling for rlim_cur) */
};

int getrlimit(int resource, struct rlimit* rlim);
int setrlimit(int resource, const struct rlimit* rlim);
\end{cminted}
  
  \subsection{Механизмы изоляции}
    \begin{itemize}
      \item \mintinline{bash}{man 7 namespaces} и \mintinline{bash}{man 7 cgroups}
      \item Linux namespaces изолируют части отдельные процессов
      \item CGroups (control groups) обычно ограничивают потребляемое процессорное время и память
      \item Существующие неймспейсы: cgroup, IPC, network, mount, PID, time, UTS
    \end{itemize}
