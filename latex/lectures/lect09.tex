\newpage
\section{Сети, часть 2}
\begin{figure}[H]
  \centering
  \includegraphics[width=0.5\linewidth]{/Users/user/Downloads/nat_problem.png}
  \caption{NAT problem}
  \label{fig:nat_problem}
\end{figure}

  \subsection{NAT}
    \begin{itemize}
      \item Network address translation
      \item По определённым правилам преобразует source/destination в IP-пакете (SNAT/DNAT)
    \end{itemize}
    
\begin{figure}[H]
  \centering
  \includegraphics[width=0.8\linewidth]{/Users/user/Downloads/nat_example.png}
  \caption{NAT example}
  \label{fig:nat_example}
\end{figure}
  
  \subsection{Full-cone NAT}
    \begin{itemize}
      \item \cmint{(INT_ADDR, INT_PORT) <=> (EXTERNAL_ADDR, INT_PORT, *, *)}
      \item В обратную сторону пакеты будут направляться от любых пар адрес-порт
      \item Port forwarding
    \end{itemize}

  \subsection{(Address-)restricted-cone NAT}
    \begin{itemize}
      \item \cmint{(INT_ADDR, INT_PORT) <=> (EXTERNAL_ADDR, INT_PORT, REM_ADDR, *)}
      \item В обратную сторону пакеты будут направляться, если клиент уже открывал соединение до сервера  
    \end{itemize}
  
  \subsection{Port-restricted-cone NAT}
    \begin{itemize}
      \item \cmint{(INT_ADDR, INT_PORT) <=> (EXTERNAL_ADDR, INT_PORT, REM_ADDR, REM_PORT)}
      \item В обратную сторону пакеты будут направляться, если клиент уже открывал соединение до сервера до определённого порта
    \end{itemize}
  
  \subsection{Symmetric NAT}
    \begin{itemize}
      \item \cmint{(INT_ADDR, INT_PORT) <=> (EXT_ADDR, EXT_PORT, REM_ADDR, REM_PORT)}
      \item \cmint{EXT_PORT} выбирается случайным образом
    \end{itemize}
  
  \subsection{NAT traversal}
    \begin{itemize}
      \item Проблема: два клиента за NAT хотят установить соединение, как это сделать?
      \item Для этого используется внешний сервер с общедоступным адресом, который определяет типы NAT'ов
      \item В зависимости от типов NAT, можно сделать hole punching или нельзя (например, symmetric + symmetric)
      \item Если NAT traversal возможен (full-cone + restricted-cone), то используется STUN
      \item Если NAT traversal сделать невозможно, используются выделенные сервера-прокси (TURN)
    \end{itemize}
    
    \href{https://tailscale.com/blog/how-nat-traversal-works/}{Невероятно подробная статья про NAT}
  
  \subsection{Как это устроено в Linux: devices}
    \begin{itemize}
      \item Интерфейс или устройство (interface/device) — сущность в ядре, обычно: сетевая карта или loopback интерфейс
      \item Устройство имеет свой уникальный link-layer address (MAC) \cmint{ip link}
      \item Каждое устройство может также иметь несколько network адресов (IP + маска подсети) \cmint{ip addr}
    \end{itemize}
  
  \subsection{Как это устроено в Linux: routing tables}
    \begin{itemize}
      \item Несколько таблиц маршрутизации (пользователь может задавать вплоть до 252 таблиц)
      \item \cmint{ip route show table <table name>}
      \item Две встроенные: local и main
      \item local описывает локальные адреса интерфейсов (которые приземляются на этом же хосте)
      \item main описывает основную таблицу маршрутизации \cmint{ip rule}
    \end{itemize}
  
  \subsection{DHCP}
    \begin{itemize}
      \item Адреса интерфейсов можно назначать вручную, однако это неудобно
      \item DHCP = Dynamic Host Configuration Protocol
      \item В IPv6 DHCP заменён SLAAC (Stateless address autoconfiguration)
    \end{itemize}
  
  \subsection{DHCP: DORA}
\begin{figure}[H]
  \centering
  \includegraphics[width=1\linewidth]{/Users/user/Downloads/dhcp_flow.png}
  \caption{DHCP flow}
  \label{fig:dhcp}
\end{figure}

  \subsection{Netfilter}
    \begin{itemize}
      \item Подсистема Linux для перехвата, мониторига и изменения пакетов сетевого стека
      \item Состоит из правил, которые разбиты на таблицы и цепочки (chains)
    \end{itemize}
    
\begin{figure}[H]
  \centering
  \includegraphics[height=0.6\textheight]{/Users/user/Downloads/netfilter_flow.jpg}
  \caption{Netfilter flow}
  \label{fig:netfilter}
\end{figure}

  \subsection{Таблицы Netfilter}
    \subsubsection{conntrack}
      \begin{itemize}
        \item Stateful firewall
        \item Позволяет заглядывать внутрь TCP-соединений
        \item Например, можно парсить первые несколько байт соединения и оверрайдить dstPort в таких пакетах
      \end{itemize}
    
    \subsubsection{mangle}
      \begin{itemize}
        \item Таблица, в которой можно менять некоторые поля в заголовке IP-пакета
        \item Например, TTL или помечать специальной меткой (MARK), чтобы в дальнейших правилах её учитывать
      \end{itemize}
    
    \subsubsection{nat}
      \begin{itemize}
        \item Таблица для SNAT/DNAT/MASQUERADE правил
      \end{itemize}
    
    \subsubsection{filter}
      \begin{itemize}
        \item Таблица для фильтрации пакетов
        \item Здесь можно, например, просто выбросить пакет, который пришёл от определённого адреса
      \end{itemize}
  
  \subsection{\cmint{iptables}}
    \begin{itemize}
      \item CLI для управления netfilter
      \item \cmint{iptables -A INPUT -s 8.8.8.8 -j DROP}
      \item \cmint{iptables -t nat -A POSTROUTING -o eth0 -j SNAT --to 192.168.1.1}
      \item Более современные аналоги: \cmint{nftables}, \cmint{ufw}
    \end{itemize}
    \href{https://stackoverflow.com/questions/11719572/how-to-simulate-different-nat-behaviours}{Как создать разные типы NAT с помощью iptables}
  
  \subsection{iptables под капотом: ioctl vs AF\_NETLINK}
    \begin{itemize}
      \item Устаревший способ: \cmint{ioctl} для конфигурирования сети (\bmint{man 7 netdevice})
      \item Современный способ: семейство протоколов \cmint{AF_NETLINK}
      \item \cmint{fd = socket(AF_NETLINK, socket_type, netlink_family);}
      \item \cmint{netlink_family} может быть разным:
      \item \cmint{NETLINK_ROUTE} для замены netdevice (\bmint{man 7 rtnetlink})
      \item \cmint{NETLINK_NETFILTER} для операций с netfilter
    \end{itemize}
  
  \subsection{Туннелирование}
    \begin{itemize}
      \item Создание «туннеля» между двумя точками через другие сети
      \item Пакет, который проходит через туннель, помещается (инкапуслируется) в другой пакет, который уже идёт по сети
      \item Примеры: IPIP туннелирование, GRE (generic routing encapsulation)
    \end{itemize}
  
  \subsection{Userspace tunneling}
    \begin{itemize}
      \item Для туннелирования обычно нужно написать свой модуль в ядро
      \item TUN/TAP interfaces
      \item Пакеты отправляются в tun/tap-интерфейс, оттуда пользовательский процесс их получает и отправляет по сети через другой интерфейс, на другом действия выполняются в обратном порядке
    \end{itemize}
    
\begin{figure}[H]
  \centering
  \includegraphics[width=1\linewidth]{/Users/user/Downloads/tunneling.png}
  \caption{Tunneling}
  \label{fig:tunneling}
\end{figure}
