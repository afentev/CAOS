\newpage
\section{Ассемблер x86}
  \subsection{Програмное управление}
    \begin{itemize}
      \item Программа и данные хранятся в оперативной памяти
      \item Программа кодируется в виде инструкций процессора
      \item Шаги выполнения инструкции:
      \begin{enumerate}
        \item Instruction fetch
        \item Instruction decode and register fetch
        \item Execute
        \item Memory access
        \item Register write back
      \end{enumerate}
    \end{itemize}
  
  \subsection{Ассемблер}
    \begin{itemize}
      \item Максимально приближен к машинному коду
      \item Вследствие этого платформозависимый
    \end{itemize}
  
  \subsection{CISC vs RISC}
    \begin{itemize}
      \item Complex(Reduced) Instruction Set Computers
      \item x86 CISC, ARM RISC
      \item Переменная длина инструкции и больше набор инструкций
      \item Сложности с пайпланингом
      \item Инструкции исполняются разное число тактов
    \end{itemize}
  
  \subsection{Регистры}
    \begin{itemize}
      \item Очень быстрые ячейки памяти на самом процессоре
      \item по 64 бита на x86-64
      \item числа в two-complement little endian
      \item Регистры общего назначения: \cmint{rax}, \cmint{rbx}, \cmint{rcx}, \cmint{rdx}, \cmint{rsi}, \cmint{rdi}, \cmint{r8}-\cmint{r15}
      \item Специальные: \cmint{rbp}, \cmint{rsp}, \cmint{rip}, \cmint{rflags}, ...
    \end{itemize}
  
  \subsubsection{Вложенность регистров}
\begin{figure}[H]
  \centering
  \includegraphics[width=1\linewidth]{/Users/user/Downloads/register.png}
  \caption{Registers}
  \label{fig:registers}
\end{figure}

  \subsection{Команды}
    \begin{asmminted}
label_name:
  cmd op1, op2, ...  
    \end{asmminted}
    
    Типы операндов команд:
    \begin{itemize}
      \item \cmint{imm} — константы
      \item \cmint{r} — регистры
      \item \cmint{m} — память
    \end{itemize}
    Хотя бы один из операндов должен быть регистром.
  
  \subsection{Intel vs AT\&T синтаксисы}
    \begin{itemize}
      \item Intel: \asmmint{mov rax, [rax, 2 * rcx + 0x10]}
      \item AT\&T: \bmint{mov 0x10(%rax, %rcx, 2), %rax}
    \end{itemize}
  
  \subsection{Работа с памятью}
    \begin{itemize}
      \item \asmmint{mov} загружает значение ячейки памяти
      \item \asmmint{lea} загружает адрес ячейки памяти
    \end{itemize}
    Например: \asmmint{mov rax, [rsi]}
  
  \subsection{Арифметика}
    \begin{itemize}
      \item \asmmint{add}
      \item \asmmint{sub}
      \item \asmmint{mul}
      \item \asmmint{div}
    \end{itemize}
    
    Например: \asmmint{add dst, src}

  \subsection{Флаги}
    \begin{itemize}
      \item Регистр EFLAGS содержит специальные биты-флаги результата операции
      \item Основные флаги:
      \begin{enumerate}
        \item \asmmint{ZF} в результате операции получился 0
        \item \asmmint{SF} результат отрицательный
        \item \asmmint{OF} знаковое переполнение
        \item \asmmint{CF} беззнаковое переполнение
      \end{enumerate}
    \end{itemize}
  
  \subsubsection{Примеры выставления флагов}
    Для простоты будем рассуждать в терминах 4-битных чисел:
    \begin{itemize}
      \item \bmint{0x0001 - 0x0001 = 0x0000} (выставится \asmmint{ZF})
      \item \bmint{0x0000 - 0x0001 = 0x1111} (выставится \asmmint{SF})
      \item \bmint{0x1111 + 0x0001 = 0x0000} (выставится \asmmint{CF})
      \item \bmint{0x0111 + 0x0001 = 0x1000} (выставится \asmmint{OF})
    \end{itemize}
    
  \subsubsection{Вычисление флагов}
    \begin{itemize}
      \item Есть инструкции для вычисления флагов без изменения регистров общего назначения
      \item Результат используется для условных переходов
      \item \asmmint{cmp} (аналог \asmmint{sub})
      \item \asmmint{test} (аналог \asmmint{and})
    \end{itemize}
  
  \subsection{Переходы}
    \begin{itemize}
      \item \asmmint{j**} label\_name перейти на метку
      \item \asmmint{jmp} безусловный переход
      \item \asmmint{jz} перейти, если выставлен \asmmint{ZF}
      \item Для знакомых чисел \asmmint{jl}, \asmmint{jle} и т. д.
      \item Для беззнаковых чисел \asmmint{jb}, \asmmint{jbe} и т. д.
    \end{itemize}
  
  \subsection{Godbolt}
\begin{figure}[H]
  \centering
  \includegraphics[width=1\linewidth]{/Users/user/Downloads/godbolt.png}
  \caption{Godbolt}
  \label{fig:godbolt}
\end{figure}

  \subsection{Calling conventions}
    \begin{itemize}
      \item \asmmint{rsp} указывает на вершину стека
      \item Первые 6 аргументов передаются в регистрах \asmmint{rdi}, \asmmint{rsi}, \asmmint{rdx}, \asmmint{rcx}, \asmmint{r8}, \asmmint{r9}, остальные на стеке
      \item Возвращается значение в \asmmint{rax}
      \item При вызове функции через \asmmint{call} на вершину кладётся адрес возврата
      \item \asmmint{ret} достаёт со стека адрес возврата и переходит по нему
    \end{itemize}
    
\begin{figure}[H]
  \centering
  \includegraphics[width=0.6\linewidth]{/Users/user/Downloads/stack_x64.png}
  \caption{Stack}
  \label{fig:stack}
\end{figure}

    \begin{itemize}
      \item При вызове функций вершин стека (\asmmint{rsp}) должна быть выровнена по 16 байт
      \item Нужно сохранить значения \asmmint{rax}, \asmmint{rdi}, \asmmint{rsi}, \asmmint{rdx}, \asmmint{rcx}, \asmmint{r8}, \asmmint{r9}, \asmmint{r10}, \asmmint{r11}, если они вам нужны
      \item \asmmint{rbx}, \asmmint{rsp}, \asmmint{rbp}, \asmmint{r12}, \asmmint{r13}, \asmmint{r14}, \asmmint{r15} нужно вернуть в неизменном виде после выхода из функции
    \end{itemize}
  
  \subsection{Векторные регистры}
    \begin{itemize}
      \item SIMD (single instruction multiple data)
      \item SSE/AVX/AVX-512
      \item \asmmint{xmm0}-\asmmint{xmm15}, \asmmint{ymm0}-\asmmint{ymm15}, \asmmint{zmm0}-\asmmint{zmm15}, caller-saved
      \item Аргументы передаются в \asmmint{xmm0}-\asmmint{xmm7}
      \item Возвращаемое значение в \asmmint{xmm0}
    \end{itemize}
  
  \subsection{Скалярные инструкции}
    \begin{itemize}
      \item Суффикс \asmmint{sd} означает \asmmint{double}, \asmmint{ss} означает \asmmint{float}
      \item Арифметика: \asmmint{addsd dst, src}
      \item Преобразование типов: \asmmint{cvtsi2sd dst, src}
      \item Сравнение: \asmmint{comisd dst, src}
    \end{itemize}
  
  \subsection{Векторные инструкции}
    Общий вид: \bmint{op[ap|up][s|d] dst, src}
    \begin{itemize}
      \item \asmmint{ap} для загрузки из памяти, выровненной по длине регистра
      \item \asmmint{s} для \asmmint{float}, \asmmint{d} для \asmmint{double}
    \end{itemize}
    Пример: \asmmint{addps xmm0, xmm1}
    Для AVX: \asmmint{vaddps xmm0, xmm1, xmm2}
  
  \subsubsection{Intrinsincs}
    Расширения компилятора, позволяющие использовать векторные инструкции в коде на C.
    \begin{cminted}
for (int i = 0; i < n; i+= 8) {
  __m256 r1 = _mm256_load_ps(a + i);
  __m256 r2 = _mm256_load_ps(b + i);
  __m256 r3 = _mm256_add_ps(r1, r2);
  _mm256_store_ps(&c[i], r3);
}  
    \end{cminted}
