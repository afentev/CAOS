\section{Файлы}
  \subsection{Файлы и директории}
    Файл -- это сущность, которая содержит данные и имеет имя.\\
    В Unix: Everything is a file!
  
  \subsubsection{Имя файла}
    \begin{itemize}
      \item Не более PATH\_MAX символов: 4 Кб на современных ОС, 256 байт для portability
      \item PATH\_MAX включает \textbackslash 0 в конце
      \item Разделитель пути -- /
      \item Части пути не более 255 символов каждая
    \end{itemize}
    
  \subsubsection{Имя директории}
    \begin{itemize}
      \item Абсолютный путь: начинается с корня (например, /Users/carzil/mipt)
      \item Относительный путь: вычисляется от текущей директории (например, carzil/mipt)
      \item . -- текущая директория (./carzil/mipt = carzil/mipt и ./carzil/./././mipt = ./carzil/mipt)
      \item .. -- директорий выше (/Users/carzil/mipt/.. = /Users/carzil)
    \end{itemize}
    
  \subsection{Файловые системы}
      \begin{itemize}
        \item Структура данных для организации хранения информации
        \item Метаданные -- информация о файле: дата последнего изменения, права доступа, создатель и так далее
        \item Работают поверх хранилища (HDD, SSD, NVMe)
        \item Хранилище традиционно разбивается на \textit{блоки}
        \item Размер блоков обычно 512 байт или 4Кб
      \end{itemize}
  
  \subsubsection{ext2}
    \begin{itemize}
      \item Linux, 1993 год
      \item inode -- физическое представление файла на диске: заголовок с метаинформацией + информация где он хранится
      \item Директории тоже хранятся в inode, так как директория -- файл!
    \end{itemize}

\begin{figure}[H]
  \centering
  \includegraphics[width=0.8\linewidth]{/Users/user/Downloads/ext2_inode.png}
  \caption{inode}
  \label{fig:inode}
\end{figure}  
  
  \subsubsection{ext4}
    \begin{itemize}
      \item 2006 год
      \item Де-факто стандартная файловая система для Linux
      \item Журналируемая
      \item Для больших директорий используется HTree
    \end{itemize}
  
    Во времена ext2 была проблема: считалось, что жесткие диски живут долго и работают безотказно, однако на самом деле это было не так. При записи файла, например, могла произойти ошибка на жестком диске, и из-за этого ext2 ломалась. По этой причине в ext3 сделали журнал. Журнал -- это область на диске, в которую записываются логи всех операций, которые выполняются с данными. В ext3 в журнал записываются блоки, которые мы меняем на диске. Если в процессе изменения структуры данных файловой системы случился сбой диска, то в логи не будет записан маркер конца транзакции изменения файловой системы. И когда мы будем восстанавливать файловую систему по журналу, мы просто применим все записанные транзакции и получим какое-то её стабильное состояние. Также у нас есть возможность обратить изменения на диске. Это нам гарантирует, что если посреди операции возникла ошибка, мы не сломаем все данные, и сможем всё восстановить. Журналируемость можно отключить, благодаря этому файловая система будет работать быстрее. \\
    Еще одно отличие ext2 и ext3: есть директории, в которых находятся очень много файлов. Из-за этого поиск в таких директориях работает долго: нужно пробежаться по всем записям в inode и сравнить названия файлов с тем, что мы ищем. Поэтому в ext3 (и в ext4 соответственно) была создана структура данных HTree, которая хранит кэши файловых имен. Благодаря этому можно за амортизированное $\mathcal{O}(1)$ искать файлы в директориях. Использование HTree тоже можно отключить.
    
  \subsubsection{Другие файловые системы}
    \begin{itemize}
      \item FAT32
      \item NTFS (проприетарная, используется в Windows)
      \item ReiserFS (оптимизирует работу с большим количеством маленьких файлов)
      \item ZPS (может работать с несколькими устройствами)
    \end{itemize}
  
  \subsubsection{sysfs и procfs}
    \begin{itemize}
      \item ``Метафайловые системы''
      \item Не имеют никаких данных на диске, возвращают информацию напрямую из ядра Linux
      \item Часто используются, чтобы не добавлять новые сисколлы
    \end{itemize}

  \subsubsection{FUSE}
    \begin{itemize}
      \item Код файловой системы обычно расположен в ядре -- это неудобно
      \item FUSE = file system in userspace
    \end{itemize}
    Когда мы хотим поменять файл, мы сообщаем об этом операционной системе, а та, видя, что мы обращаемся к FUSE-файловой системе, передаст этот запрос процессу файловой системы.\\
    Пример использования: можно работать с файлами на удаленной машине, как будто они находятся локально на нашем компьютере.  В таком случае все изменения файловой системы будут пересылаться по сети. Это можно сделать при помощи SSHFS.
  
  \subsection{Файловые дескрипторы}
    \begin{itemize}
      \item Преобразование имени файла в inode -- очень дорогая операция, которая может требовать много обращений к диску
      \item Этот процесс ``кэшируют'' с помощью файловых дескрипторов
      \item Файловый дескриптор -- число больше 0
      \item Новый файловый дескриптор будет минимальным доступным числом
    \end{itemize}
  Благодаря файловым дескрипторам можно только один раз делать преобразование имени файла в inode (делая сисколл open). При этом привязка произойдет именно к inode, а не к имени файла.
    \begin{itemize}
      \item За каждым файловым дескриптором скрывается \href{https://elixir.bootlin.com/linux/v5.19.12/source/include/linux/fs.h#L925}{специальная структура} в ядре
      \item Указатель на inode, позиция в файле, флаги (чтения/запись/блокирование), различные локи и так далее
    \end{itemize}
  
    \subsubsection{Работа с данными файла}
\begin{cminted}
#include <unistd.h>

int open(const char *pathname, int flags, mode_t mode);
ssize_t read(int fd, void *buf, size_t count);
ssize_t write(int fd, const void *buf, size_t count);
int close(int fd);
\end{cminted}
    
    \subsubsection{Работа с метаданными файла}
\begin{cminted}
#include <sys/stat.h>

int stat(const char* path, struct stat* buf);
int fstat(int fd, struct stat *statbuf);
int lstat(const char* pathname, struct stat* statbuf);

struct stat {
  dev_t     st_dev;
  ino_t     st_ino;
  mode_t    st_mode;
  nlink_t   st_nlink;
  uid_t     st_uid;
  gid_t     st_gid;
  dev_t     st_rdev;
  off_t     st_size;
  blksize_t st_blksize;
  blkcnt_t  st_blocks;
  struct timespec st_atime/st_mtime/st_ctime;
};
\end{cminted}
    
    \subsubsection{Права доступа}
      \begin{itemize}
        \item rwx = \textbf{R}ead/\textbf{W}rite/e\textbf{X}ecute
        \item 9 бит, 3 группы; права владельца, права группы и права для остальных
        \item Часто записываются как числа в восьмиричной системе счисления
        \item $777_8 = 111111111_2$ = \bmint{rwxrwxrwx}
        \item $664_8 = 110100100_2$ = \bmint{rw-r--r--}
      \end{itemize}
    
    \subsubsection{Права доступа для директорий}
      \begin{itemize}
        \item r -- листинг директории
        \item w -- создание файлов внутри директории
        \item x -- возможность перейти в директорию (cd),  а также доступ к файлам
      \end{itemize}
    
    \subsubsection{Регулярные файлы}
      \begin{itemize}
        \item \cmint{S_ISREG(stat.st_mode)}
        \item Обычные файлы с данными
      \end{itemize}
    
    \subsubsection{Директории}
      \begin{itemize}
        \item \cmint{S_ISDIR(stat.st_mode)}
        \item Специальный API для чтения, обычные read/write не работают
        \item Создание и удаление: mkdir/rmdir
      \end{itemize}
    
\begin{cminted}
#include <dirent.h>

struct dirent *readdir(DIR *dirp);

struct dirent {
  ino_t          d_ino;
  off_t          d_off;
  unsigned short d_reclen;
  unsigned char  d_type;
  char           d_name[256];
};

int closedir(DIR *dirp);
\end{cminted}
    
    \subsubsection{Символические ссылки}
      \begin{itemize}
        \item \cmint{S_ISLINK(stat.st_mode)}
        \item Аналог \cppmint{std::weak_ptr} для inode
        \item Могут быть dangling: то есть ссылаться на файл, которого нет
        \item Отдельный тип файла
        \item Путь, на который она ссылается, записан в блоках
      \end{itemize}
    
    \subsubsection{Жесткие ссылки}
      \begin{itemize}
        \item Аналог \cppmint{std::shared_ptr} для inode
        \item Только внутри одной файловой системы
        \item Если количество жестких ссылок стало равно 0, то inode становится свободной
        \item Не файл, а сущность файловой системы
      \end{itemize}
    
    \subsubsection{Символьные устройства (character device)}
      \begin{itemize}
        \item \cmint{S_ISCHR(stat.st_mode)}
        \item Устройства, из которых можно последовательно читать
        \item Клавиатура, звуковая карта, сетевая карта
        \item Такие файлы создаются драйверами ядра
      \end{itemize}
    
    \subsubsection{Блочные устройства (block device)}
      \begin{itemize}
        \item \cmint{S_ISBLC(stat.st_mode)}
        \item Разбиты на блоки одинакового размера
        \item Можно прочитать любой блок
        \item HDD, SSD, NAS
      \end{itemize}