\subsection{ELF}
  \begin{itemize}
    \item Executable \& Linkable Format
    \item Исполняемый формат файлов для Linux
    \item Поддерживает разные архитектуры и битности
    \item \href{http://refspecs.linuxbase.org/elf/x86_64-abi-0.99.pdf}{Спецификация}
    \item \href{https://github.com/torvalds/linux/blob/master/include/uapi/linux/elf.h}{Структуры внутри ядра}
  \end{itemize}

\begin{figure}[H]
\centering
  \includegraphics[width=0.7\linewidth]{/Users/user/Downloads/elf-structure.png}
  \caption{ELF structure}
  \label{fig:elf_structure}
\end{figure} 
  
\subsection{ELF: header}

\begin{cminted}
typedef struct elf64_hdr {
  unsigned char  e_ident[EI_NIDENT];  /* ELF "magic number" */
  Elf64_Half e_type;
  Elf64_Half e_machine;
  Elf64_Word e_version;
  Elf64_Addr e_entry;    /* Entry point virtual address */
  Elf64_Off e_phoff;    /* Program header table file offset */
  Elf64_Off e_shoff;    /* Section header table file offset */
  Elf64_Word e_flags;
  Elf64_Half e_ehsize;
  Elf64_Half e_phentsize;
  Elf64_Half e_phnum;
  Elf64_Half e_shentsize;
  Elf64_Half e_shnum;
  Elf64_Half e_shstrndx;
} Elf64_Ehdr;
\end{cminted}

\subsubsection{ELF: секции}
  \begin{itemize}
    \item Секции хранят непосрдественно данные
    \item Каждая секция имеет своё имя
    \item Перечень всех секций хранит \textit{таблица секций}
  \end{itemize}
    
  \begin{itemize}
    \item .data -- данные
    \item .text -- исполняемый код
    \item .rodata -- read-only данные
    \item .symtab -- таблица символов
    \item .strtab -- таблица строк
    \item .shstrtab -- таблица строк с названием секций
    \item .rel/.rela -- таблица релокаций
  \end{itemize}
  
\subsubsection{ELF: .symtab}
  \begin{cminted}
typedef struct {
    uint32_t      st_name;
    unsigned char st_info;
    unsigned char st_other;
    uint16_t      st_shndx;
    Elf64_Addr    st_value;
    uint64_t      st_size;
} Elf64_Sym;
  \end{cminted}
  
\subsubsection{ELF: сегменты}
  \begin{itemize}
    \item Все секции объединяются в \textit{сегменты}
    \item Каждый сегмент имеет адрес, по которому он загружается
    \item Program header array содержит все сегменты, располагается в начале файла
  \end{itemize}
  
\subsubsection{ELF: program header}
  \begin{cminted}
typedef struct elf64_phdr {
  Elf64_Word p_type;
  Elf64_Word p_flags;
  Elf64_Off p_offset;    /* Segment file offset */
  Elf64_Addr p_vaddr;    /* Segment virtual address */
  Elf64_Addr p_paddr;    /* Segment physical address */
  Elf64_Xword p_filesz;    /* Segment size in file */
  Elf64_Xword p_memsz;    /* Segment size in memory */
  Elf64_Xword p_align;    /* Segment alignment, file & memory */
} Elf64_Phdr;
  \end{cminted}

Разница между \cmint{p_memsz} и \cmint{p_filesz} существует и обуславливается наличием сегмента .bss, в котором хранятся статические переменные, инициализирируемые нулями. Такие переменные не хранятся на диске, но при запуске программы место под них отводится. Поэтому \cmint{p_filesz} -- это размер на диске, а \cmint{p_memsz} -- размер загружаемого файла.

\subsection{execve}
  \begin{itemize}
    \item Парсит первые несколько байт файла
    \item Ищет ELF magic или shebang
    \item Загружает образ (mmap'ит)
    \item Подготавливает окружение для старта процесса (стек, переменные окружение, etc)
    \item Запускает инструкцию по адресу \cmint{e_entry}
  \end{itemize}
  
\section{Linux scheduler}
  \subsection{Realtime scheduling}
    \begin{itemize}
      \item У каждого процесса есть собственный realtime приоритет
      \item Планировщик ищет процесс с наибольшим приоритетом и запускает его, пока он не выйдет
      \item Вытеснения нет!
      \item Round-robin для процессов с одним приоритетом
    \end{itemize}
  
  \subsection{Process niceness}
    \begin{itemize}
      \item Значения от -20 до +19
      \item Процесс с меньшим niceness должен получать больше процессорного времени
    \end{itemize}
  
  \subsubsection{Timeslice}
    \begin{itemize}
      \item Квант процессорного времени
      \item В зависимости от типа системы от 1мс до 10-20мс
      \item CPU bound процессы: в основном выполняют инструкции
      \item I/O bound процессы: в основном спят
    \end{itemize}
  
  \subsubsection{Timeslice \& niceness}
    \begin{itemize}
      \item Зададим каждому приоритету определённый timeslice (например, 0 = 100ms, +20 = 5ms)
      \item Процессы с меньшим NI получают больше процессорного времени
    \end{itemize}
  
  \subsubsection{Timeslice \& niceness: проблемы подхода}
    \begin{itemize}
      \item Два процесса с NI = +19 и два процесса с NI = 0
      \item Первые получают по 5ms процессорного времени
      \item Вторые по 100ms
      \item В обоих случаях 50\% CPU, но переключение контекста у первых происходит чаще
    \end{itemize}
  
    И другие проблемы:
    \begin{itemize}
      \item Два процесса NI = 0 и NI = 1 vs два процесса NI = +18 и NI = +19
      \item \bmint{5ms/10ms} vs \bmint{95ms/105ms}
      \item Нелинейность относительно стартового значения
    \end{itemize}
  
  \subsection{completely Fair Scheduler}
    \begin{itemize}
      \item Эмулирует идеальный multitasking процессор
      \item Идея -- раздавать процессам время пропорциональное 1/N, где N -- количество запущенных процессов
      \item Каждому процессу начисляется \cmint{vruntime} -- количество затраченного CPU времени за определенный интервал времени
      \item Очередным берётся процесс с наименьшим \cmint{vruntime}
    \end{itemize}
  
  \subsubsection{CFS: проблемы}
    \begin{itemize}
      \item Минимальное время (гранулярность) $\Rightarrow$ чем больше процессов, тем менее честным становится CFS
      \item Context switch считается zero-time операцией, но по факту это не так
    \end{itemize}

