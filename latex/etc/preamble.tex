%%% Работа с русским языком
\usepackage{cmap}					          % поиск в PDF
\usepackage[T2A]{fontenc}			      % кодировка
\usepackage[utf8]{inputenc}               % кодировка исходного текста
\usepackage[english, russian]{babel}   % локализация и переносы


%%% Страница 
\usepackage{extsizes} % Возможность сделать 14-й шрифт
\usepackage{geometry}  
\geometry{left=20mm,right=20mm,top=25mm,bottom=30mm} % задание полей текста

\usepackage{titleps}      % колонтитулы
\newpagestyle{main}{
	\setheadrule{.4pt}                      
	\sethead{\CourseName}{}{\hyperlink{intro}{\;Назад к содержанию}}
	\setfootrule{.4pt}                       
	\setfoot{\CourseDate \; ФПМИ МФТИ}{}{\thepage} 
}      
\pagestyle{main}    % Устанавливает контитулы на странице


%%%  Текст
\setlength\parindent{0pt}         % Устанавливает длину красной строки 0pt
\sloppy                                        % строго соблюдать границы текста
\linespread{1.3}                           % коэффициент межстрочного интервала
\setlength{\parskip}{0.5em}                % вертик. интервал между абзацами
%\setcounter{secnumdepth}{0}                % отключение нумерации разделов
%\setcounter{section}{-1}         % Чтобы сделать нумерацию лекций с нуля
\usepackage{multicol}				          % Для текста в нескольких колонках
%\usepackage{soul}
\usepackage{soulutf8} % Модификаторы начертания


%%% Гиппер ссылки
\usepackage{hyperref}
\usepackage[usenames,dvipsnames,svgnames,table,rgb]{xcolor}
\hypersetup{				% Гиперссылки
	unicode=true,           % русские буквы в раздела PDF\\
	pdfstartview=FitH,
	pdftitle={Заголовок},   % Заголовок
	pdfauthor={Автор},      % Автор
	pdfsubject={Тема},      % Тема
	pdfcreator={Создатель}, % Создатель
	pdfproducer={Производитель}, % Производитель
	pdfkeywords={keyword1} {key2} {key3}, % Ключевые слова
	colorlinks=true,       	% false: ссылки в рамках; true: цветные ссылки
	linkcolor=blue,          % внутренние ссылки
	citecolor=green,        % на библиографию
	filecolor=magenta,      % на файлы
	urlcolor=NavyBlue,           % на URL
}

\usepackage{todonotes}


%%% Для формул
\usepackage{amsmath}          
\usepackage{amssymb}


%%%%%% theorems
\usepackage{amsthm}  % for theoremstyle

\theoremstyle{plain} % Это стиль по умолчанию, его можно не переопределять.
\newtheorem{theorem}{Теорема}[section]
\newtheorem{prop}[theorem]{Утверждение}
\newtheorem{lemma}{Лемма}[section]
\newtheorem{sug}{Предположение}[section]

\theoremstyle{definition} % "Определение"
\newtheorem{Def}{Определение}
\newtheorem{corollary}{Следствие}[theorem]
\newtheorem{problem}{Задача}[section]

\theoremstyle{remark} % "Примечание"
\newtheorem*{nonum}{Решение}
\newtheorem*{definition}{Def}
\newtheorem*{example}{Пример}
\newtheorem*{note}{Замечание}


%%% Работа с картинками
\usepackage{graphicx}                           % Для вставки рисунков
\graphicspath{{images/}{images2/}}        % папки с картинками
\setlength\fboxsep{3pt}                    % Отступ рамки \fbox{} от рисунка
\setlength\fboxrule{1pt}                    % Толщина линий рамки \fbox{}
\usepackage{wrapfig}                     % Обтекание рисунков текстом
\graphicspath{{images/}}                     % Путь к папке с картинками

\newcommand{\drawsome}[1]{            % Для быстрой вставки картинок
    \begin{figure}[h!]
            \centering
            \includegraphics[scale=0.7]{#1}
            \label{fig:first}
    \end{figure}
}
\newcommand{\drawsomemedium}[1]{
    \begin{figure}[h!]
            \centering
            \includegraphics[scale=0.45]{#1}
            \label{fig:first}
    \end{figure}
}
\newcommand{\drawsomesmall}[1]{
    \begin{figure}[h!]
            \centering
            \includegraphics[scale=0.3]{#1}
            \label{fig:first}
    \end{figure}
}


%%% облегчение математических обозначений
\newcommand{\R}{\mathbb{R}}
\newcommand{\N}{\mathbb{N}}
%\newcommand{\C}{\mathbb{C}}             % команда уже определена где-то)
\newcommand{\Z}{\mathbb{Z}}
\newcommand{\E}{\mathbb{E}}
\newcommand{\brackets}[1]{\left({#1}\right)}      % автоматический размер скобочек
% Здесь можно добавить ваши индивидуальные сокращения 

%%%%%%%%%%%%%%%%%%%%%%%%%%%

\usepackage[utf8]{inputenc} % Включаем поддержку UTF8
\usepackage[russian]{babel}  % Включаем пакет для поддержки русского языка
\usepackage{mathtools}
\usepackage{amssymb}
\usepackage{amsthm}
\usepackage{graphicx}
\usepackage{subcaption}\usepackage{caption}
\usepackage{svg}

\usepackage{listings}
\usepackage{xcolor}

\newcommand{\listingsttfamily}{\fontfamily{NotoSansMono-TLF}\big}
\newcommand*\lsin{\lstinline[columns=fixed, basicstyle=\fontsize{13}{13}\ttfamily]}

\lstset{
  language = [11]C++,
  backgroundcolor=\color{black!5}, % set backgroundcolor
  keywordstyle = \color{blue},
  columns = flexible
}

% define some basic colors
\definecolor{mauve}{rgb}{0.58,0,0.82}

\lstdefinestyle{asm}{
    belowcaptionskip=1\baselineskip,
    frame=single, 
    frameround=tttt,
    xleftmargin=\parindent,
    language=[x86masm]Assembler,
    basicstyle=\footnotesize\ttfamily,
    commentstyle=\itshape\color{green!60!black},
    keywordstyle=\color{blue!80!black},
    identifierstyle=\color{red!80!black},
    tabsize=4,
    numbers=left,
    numbersep=8pt,
    stepnumber=1,
    numberstyle=\tiny\color{gray}, 
    columns = fullflexible,
    basicstyle=\fontsize{13}{13}\ttfamily
}

\lstdefinestyle{cpp}{
    belowcaptionskip=1\baselineskip,
    frame=single, 
    frameround=tttt,
    xleftmargin=\parindent,
    language=[11]C++,
    basicstyle=\footnotesize\ttfamily,
    commentstyle=\itshape\color{green!60!black},
    keywordstyle=\color{blue!80!black},
    identifierstyle=\color{red!80!black},
    tabsize=4,
    numbers=left,
    numbersep=8pt,
    stepnumber=1,
    numberstyle=\tiny\color{gray}, 
    columns = fullflexible,
    basicstyle=\fontsize{13}{13}\ttfamily
}

\lstset{
  basicstyle=\fontsize{13}{13}\ttfamily,
  breaklines=true,                              % break long lines
  commentstyle=\itshape\color{green!50!black},  % comments are green
  keywordstyle=[1]\color{blue!80!black},        % instructions are blue
  keywordstyle=[2]\color{orange!80!black},      % sections/other directives are orange
  keywordstyle=[3]\color{red!50!black},         % registers are red
  stringstyle=\color{mauve},                    % strings are from the telekom
  identifierstyle=\color{teal},                 % user declared addresses are teal
  frame=l,                                      % black line on the left side of code
  language=[x86masm]Assembler,                   % all code is RISC-V
  tabsize=4,                                    % indent tabs with 4 spaces
  showstringspaces=false                        % do not replace spaces with weird underlines,
}
                  
\newcommand\round[1]{\left[#1\right]}

\tolerance=1
\emergencystretch=\maxdimen
\hyphenpenalty=10000
\hbadness=10000
